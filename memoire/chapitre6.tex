\chapter{Estimation des paramètres de la distribution de Laplace
  asymétrique généralisée}

Ce chapitre présente les différents éléments qui permettent d'obtenir
des estimateurs convergents à l'aide de la méthode des moments
généralisée et de l'équation d'estimation optimale. Premièrement, une
méthode pour obtenir un point de départ efficace pour l'algorithme
d'optimisation sera détaillée. Enfin, les résultats découlant des deux
chapitres précédents seront présentés.

\section{Vecteur de paramètres initiaux}
\label{sec:condinitGAL}

Comme chacune des deux méthodes d'estimation étudiées requiert
l'utilisation d'un algorithme d'optimisation numérique, on doit être
en mesure de fournir un vecteur de paramètres initiaux qui favorisera
la convergence de celui-ci. Ce vecteur doit faire partie de l'espace
des paramètres $\Omega$, tel que défini à la table
\ref{tab:roleparamGAL}. Ce vecteur est plus facile à obtenir de
manière empirique, par exemple, avec la méthode des moments, lorsque
c'est possible. Cependant, pour la distribution de Laplace asymétrique
généralisée, le système d'équations formé à partir des quatre premiers
moments centraux empiriques et théoriques \eqref{eq:momentsGAL} n'a
pas de solution analytique.

\cite{seneta2004fitting} propose une technique pour obtenir un vecteur
de paramètres initiaux basé sur la méthode des moments. En utilisant
la même démarche que ce dernier, on peut obtenir un vecteur pour la
forme en $\mu$ de la fonction caractéristique
\eqref{eq:fncaractGALmu}. On pose comme hypothèse dans les équations
des quatre premiers moments \eqref{eq:moments1GAL},
\eqref{eq:moments2GAL}, \eqref{eq:moments5GAL} et
\eqref{eq:moments6GAL}, que le paramètre $\mu$ est significativement
nul, donc que la distribution est presque symétrique. On impose donc
que les puissances de ce paramètre plus grandes ou égales à deux
soient nulles:
\begin{align}
  \label{eq:hypotheseMoM}
  \mu^k=0, \quad k\in\{2,3,4\}.
\end{align}

En posant l'égalité entre les moments théoriques et leur estimateur
correspondant, on obtient le système d'équations suivant:
\begin{align*}
  \hat{m_1} &= \theta+\tau\,\mu \\
  \hat{m_2} &= {\sigma}^{2}\,\tau \\
  \hat{\gamma_1} &= \frac{3\,\mu}{\left| \sigma\right| \,\sqrt{\tau}} \\
  \hat{\gamma_2} &= \frac{3}{\tau}.
\end{align*}

Ce système possède deux solutions analytiques, dont une respecte la
condition selon laquelle le paramètre $\sigma$ doit être positif. On
note que ce résultat ne constitue qu'un point de départ pour
l'algorithme d'optimisation numérique et qu'on ne peut, en aucun cas,
utiliser celui-ci à des fins statistiques. C'est un estimateur biaisé
et non convergent étant donné la condition \eqref{eq:hypotheseMoM}:
\begin{subequations}\label{eq:ptdepartGAL}
  \begin{align}
    \hat\theta &=-\frac{\hat{\gamma_1}\,\sqrt{\hat{m_2}}-\hat{\gamma_2}\,\hat{m_1}}{\hat{\gamma_2}} \label{eq:ptdepartGAL1}\\
    \hat\sigma &= \frac{\sqrt{\hat{\gamma_2}\,\hat{m_2}}}{\sqrt{3}} > 0 \label{eq:ptdepartGAL2}\\
    \hat\mu &= \frac{\hat{\gamma_1}\,\sqrt{\hat{m_2}}}{3} \label{eq:ptdepartGAL3}\\
    \hat\tau &= \frac{3}{\hat{\gamma_2}}. \label{eq:ptdepartGAL4}
  \end{align}
\end{subequations}
    
\section{Méthode des moments généralisée}

En premier lieu, on utilise la méthode des moments généralisée afin
d'estimer les paramètres de la distribution de Laplace asymétrique
généralisée. Aux fins d'illustration, seulement deux conditions de
moments seront utilisées. Cependant, pour obtenir des résultats
optimaux en termes de convergence asymptotique, on doit utiliser au
moins autant de conditions de moment que de paramètres. Ainsi, en
considérant la différence entre les moyennes et variances empirique et
théorique, on obtient les conditions de moment suivantes, qui ont une
espérance nulle sous les vrais paramètres $\theta_0$:
\begin{align}
  h(\theta;Y) &= \left[\begin{array}[]{c}
      Y - m_1 \\
      \left(Y - m_1 \right)^2 - m_2
    \end{array}\right] \label{eq:condmom}\\
  &= \left[\begin{array}[]{c}
      Y - \left(\theta+\mu\tau\right) \\
      \left(Y - \left(\theta+\mu\tau\right) \right)^2 -
      \tau\left(\sigma^2+\mu^2\right)
    \end{array}\right]. \label{eq:condmomGAL}
\end{align}

On définit la contrepartie empirique de ces conditions de moment
comme suit:
\begin{align}
  g(\theta;\mathbf{y}_T) &= \frac{1}{T} \left[\begin{array}[]{c}
      g_1(\theta;\mathbf{y}_T)\\
      g_2(\theta;\mathbf{y}_T)
    \end{array}\right] \nonumber\\
  &= \frac{1}{T} \left[\begin{array}[]{c}
      \sum_{t=1}^T y_t - m_1 \\
      \sum_{t=1}^T \left(y_t - m_1 \right)^2 - m_2
    \end{array}\right] \label{eq:condmomEMP}\\
  &= \frac{1}{T} \left[\begin{array}[]{c}
      \sum_{t=1}^Ty_t - \left(\theta+\mu\tau\right) \\
      \sum_{t=1}^T\left(y_t - \left(\theta+\mu\tau\right) \right)^2 -
      \tau\left(\sigma^2+\mu^2\right)
    \end{array}\right]. \label{eq:condmomEMPGAL}
\end{align}

On doit définir la matrice de pondération optimale.

\subsection{Matrice de pondération optimale}

On s'intéresse à la forme analytique de la matrice de pondération
optimale \eqref{eq:matricevcov1}. Celle-ci est obtenue en prenant
l'espérance du produit extérieur du vecteur des conditions de moment
théoriques sur elles-mêmes.  Pour des fins de simplification, on
définit le produit des conditions $i$ et $j$ par la notation
$H_{i,j}(\theta;Y) = h_i(\theta;Y)h_j(\theta;Y)$.
\begin{align}
  S(\theta;Y) &= E \left[\begin{array}{cc}
      H_{(1,1)}(\theta;Y) & H_{(1,2)}(\theta;Y) \\
      H_{(2,1)}(\theta;Y) & H_{(2,2)}(\theta;Y)
    \end{array} \right] \nonumber\\
  &= \left[\begin{array}{cc}
      E \left[H_{(1,1)}(\theta;Y)\right] & E \left[H_{(1,2)}(\theta;Y)\right] \\ 
      E \left[H_{(2,1)}(\theta;Y)\right] & E \left[H_{(2,2)}(\theta;Y)\right] 
    \end{array} \right]
\end{align}
où
\begin{align*}
  H_{(1,1)}(\theta;Y) &=\left( Y-\theta-\mu\,\tau\right)^{2} \\
  H_{(1,2)}(\theta;Y) &=\left( {Y}^{2}-2\,\theta\,Y-2\,\mu\,\tau\,Y+{\theta}^{2}+2\,\mu\,\tau\,\theta+{\mu}^{2}\,{\tau}^{2}-{\sigma}^{2}\,\tau-{\mu}^{2}\,\tau\right)^{2} \\
  H_{(2,2)}(\theta;Y) &=\left( Y-\theta-\mu\,\tau\right) \,\left(
    {Y}^{2}-2\,\theta\,Y-2\,\mu\,\tau\,Y+{\theta}^{2}+2\,\mu\,\tau\,\theta+{\mu}^{2}\,{\tau}^{2}-{\sigma}^{2}\,\tau-{\mu}^{2}\,\tau\right).
\end{align*}

On évalue ensuite l'espérance de chacun des éléments de la matrice, où
l'on définit de manière analogue celle du produit des conditions de
moments, $W_{(i,j)}(\theta;Y) = E[H_{(i,j)}(\theta;Y)]$.
\begin{align}
  W(\theta;Y) &= \left[\begin{array}{cc}
      W_{(1,1)}(\theta;Y) & W_{(1,2)}(\theta;Y) \\
      W_{(2,1)}(\theta;Y) & W_{(2,2)}(\theta;Y)
    \end{array} \right]
\end{align}
où
\begin{align*}
  W_{(1,1)} &= {\mu}^{2}\,{\tau}^{2}-2\,{\mu}^{2}\,\nu\,\tau+\nu\,{\sigma}^{2}+{\mu}^{2}\,{\nu}^{2}+{\mu}^{2}\,\nu \\
  W_{(2,2)} &= {\mu}^{4}\,{\tau}^{4}+\left( -2\,{\mu}^{2}\,{\sigma}^{2}-4\,{\mu}^{4}\,\nu-2\,{\mu}^{4}\right) \,{\tau}^{3}+\left( {\sigma}^{4}+\left( 10\,{\mu}^{2}\,\nu+2\,{\mu}^{2}\right) \,{\sigma}^{2}+6\,{\mu}^{4}\,{\nu}^{2}+10\,{\mu}^{4}\,\nu+{\mu}^{4}\right) \,{\tau}^{2}\\
  &+\left( -2\,\nu\,{\sigma}^{4}+ \left( -14\,{\mu}^{2}\,{\nu}^{2}-16\,{\mu}^{2}\,\nu\right) \,{\sigma}^{2}-4\,{\mu}^{4}\,{\nu}^{3}-14\,{\mu}^{4}\,{\nu}^{2}-10\,{\mu}^{4}\,\nu\right) \,\tau+\left( 3\,{\nu}^{2}+3\,\nu\right) \,{\sigma}^{4}\\
  &+\left( 6\,{\mu}^{2}\,{\nu}^{3}+18\,{\mu}^{2}\,{\nu}^{2}+12\,{\mu}^{2}\,\nu\right) \,{\sigma}^{2}+{\mu}^{4}\,{\nu}^{4}+6\,{\mu}^{4}\,{\nu}^{3}+11\,{\mu}^{4}\,{\nu}^{2}+6\,{\mu}^{4}\,\nu \\
  W_{(1,2)} &= W_{(2,1)} = 3\,{\theta}^{4}+\left( 6\,\mu\,\nu-3\right) \,{\theta}^{3}+\left( 3\,\nu\,{\sigma}^{2}+3\,{\mu}^{2}\,{\nu}^{2}+\left( 3\,{\mu}^{2}-3\,\mu\right) \,\nu\right) \,{\theta}^{2}-{\mu}^{3}\,{\tau}^{3} \\
  &+\left( \mu\,{\sigma}^{2}+3\,{\mu}^{3}\,\nu+{\mu}^{3}\right) \,{\tau}^{2}+\left( -4\,\mu\,\nu\,{\sigma}^{2}-3\,{\mu}^{3}\,{\nu}^{2}-4\,{\mu}^{3}\,\nu\right) \,\tau+\left( 3\,\mu\,{\nu}^{2}+3\,\mu\,\nu\right) \,{\sigma}^{2}\\
  &+{\mu}^{3}\,{\nu}^{3}+3\,{\mu}^{3}\,{\nu}^{2}+2\,{\mu}^{3}\,\nu.
\end{align*}

On évalue quelle serait la valeur de la variance-covariance sous des
paramètres optimaux. Ce résultat permet par la suite d'évaluer la
distribution asymptotique des estimateurs.

On obtient l'estimateur de la matrice optimale \eqref{eq:matvcovGMM}
en effectuant le produit extérieur du vecteur de conditions de moment
empirique \eqref{eq:condmomGAL}, puis en évaluant la moyenne des
matrices résultantes de l'équation \eqref{eq:matponderationproduith}:
\begin{align}
  \label{eq:matvcovGMMemp1}
  \hat{S}(\theta;\mathbf{y}_T) &= \frac{1}{T} \sum_{t=1}^T
  \left[\begin{array}{cc}
      G_{(1,1)}(\theta;y_t) & G_{(1,2)}(\theta;y_t) \\
      G_{(2,1)}(\theta;y_t) & G_{(2,2)}(\theta;y_t)
    \end{array} \right]
\end{align}
où
\begin{align*}
  G_{(1,1)}(\theta;y_t) &=\left( y_t-\theta-\mu\,\tau\right)^{2} \\
  G_{(1,2)}(\theta;y_t) &=\left( {y_t}^{2}-2\,\theta\,y_t-2\,\mu\,\tau\,y_t+{\theta}^{2}+2\,\mu\,\tau\,\theta+{\mu}^{2}\,{\tau}^{2}-{\sigma}^{2}\,\tau-{\mu}^{2}\,\tau\right)^{2} \\
  G_{(2,2)}(\theta;y_t) &=\left( y_t-\theta-\mu\,\tau\right) \,\left(
    {y_t}^{2}-2\,\theta\,y_t-2\,\mu\,\tau\,y_t+{\theta}^{2}+2\,\mu\,\tau\,\theta+{\mu}^{2}\,{\tau}^{2}-{\sigma}^{2}\,\tau-{\mu}^{2}\,\tau\right).
\end{align*}


\subsection{Variance-covariance des paramètres}

On obtient la variance-covariance asymptotique des paramètres à partir
de la variance-covariance associée aux conditions de moment, en
utilisant la méthode delta multivariée. Pour ce faire, on évalue
d'abord la valeur théorique du gradient $D(\theta)$.
\begin{align}
  D(\theta) &= E \left[ \begin{array}{cc}
      -1 & -2\,\left( Y-\theta-\mu\,\tau\right) \\
      0 & -2\,\sigma\,\tau \\
      -\tau & -2\,\tau\,\left( Y-\theta-\mu\,\tau\right) -2\,\mu\,\tau \\
      -\nu & -2\,\mu\,\left( Y-\theta-\mu\,\tau\right)
      -{\sigma}^{2}-{\mu}^{2}
    \end{array}\right] \nonumber\\
  &= \left[ \begin{array}{cc}
      -1 & 0 \\
      0 & -2\,\sigma\,\tau \\
      -\tau & -2\,\mu\,\tau \\
      -\nu & -{\sigma}^{2}-{\mu}^{2}
    \end{array}\right].
\end{align}

Ensuite, on utilise le résultat \eqref{matricevcovparamGMMnc} afin
d'obtenir la valeur de la variance-covariance $\mathcal{J}_0^{-1}$.

De la même manière, on évalue la variance-covariance des paramètres
optimaux en utilisant les valeurs empiriques de la variance-covariance
des conditions de moment \eqref{eq:matvcovGMMemp1} et du gradient
$\hat{D}(\theta,\mathbf{y}_T)$, défini par la matrice
\eqref{eq:gradientGMM}. On a essentiellement à calculer la moyenne
empirique des gradients obtenus en chaque point $y_t$, puis à utiliser
le résultat \eqref{matricevcovparamGMMnc} avec les estimateurs
$\hat{W}_T = \hat{S}^{-1}(\theta,\mathbf{y}_T)$ et
$\hat{D}(\theta,\mathbf{y}_T)$:
\begin{align}
  \hat{D}(\theta,\mathbf{y}_T) &= \frac{1}{T} \sum_{t=1}^{T}
  \left[ \begin{array}{cc}
      -1 & -2\,\left( y_t-\theta-\mu\,\tau\right) \\
      0 & -2\,\sigma\,\tau \\
      -\tau & -2\,\tau\,\left( y_t-\theta-\mu\,\tau\right) -2\,\mu\,\tau \\
      -\nu & -2\,\mu\,\left( y_t-\theta-\mu\,\tau\right)
      -{\sigma}^{2}-{\mu}^{2}
    \end{array}\right].
\end{align}

\subsection{Contraintes linéaires}

Une fois l'estimation des paramètres effectuée, on peut tester si ces
derniers pourraient en fait correspondre à un cas particulier de la
distribution de Laplace asymétrique généralisée ayant le même support,
parmi celles figurant à la table \ref{tab:casspeciauxGAL}. Pour ce
faire, on doit déterminer les paramètres de la distribution sous un
ensemble de contraintes linéaires, sous la forme
\eqref{eq:contraintelin}, correspondant au cas particulier, tel
qu'énuméré dans la table \ref{tab:contraintesGAL}.
\begin{table}[h!]
  \centering
  \begin{tabular}{ccc}
    & \multicolumn{2}{c}{\textbf{Paramètres}} \\
    \hline
    $\text{\textbf{Distribution}}$ & $R$ & $r$ \\
    \hline
    Laplace symétrique & $\begin{bmatrix}
      0&0&1&0\\
      0&0&0&1
    \end{bmatrix}$ & $\begin{bmatrix}
      0\\
      1
    \end{bmatrix}$  \\[1cm]
    Laplace asymétrique & $\begin{bmatrix} 0&0&0&1
    \end{bmatrix}$ & $\begin{bmatrix} 1
    \end{bmatrix}$  \\[1cm]
    Dégénérée à $\theta$ & $\begin{bmatrix}
      0&1&0&0\\
      0&0&1&0\\
      0&0&0&1
    \end{bmatrix}$ & $\begin{bmatrix}
      0\\
      0\\
      0\\
    \end{bmatrix}$\\\hline
  \end{tabular}
  \caption{Contraintes linéaires pour les cas particuliers de la distribution de Laplace asymétrique généralisée $GAL(\theta,\sigma,\mu,\tau)$}
  \label{tab:contraintesGAL}
\end{table}

On doit ensuite utiliser un algorithme d'optimisation numérique afin
de maximiser le lagrangien \eqref{eq:estimateurGMMlagrange} défini par
la fonction objectif $Q_T(\theta)$ \eqref{eq:objectifGMM2} et
l'ensemble de contraintes $a(\theta) = R\theta-r$ sélectionné
précédemment. Enfin, on peut effectuer les tests statistiques de la
section \ref{sec:testwald} afin de vérifier si les paramètres de la
distribution contrainte sont significativement différents de ceux de
la distribution non contrainte. Dans cette situation, l'hypothèse
selon laquelle le cas particulier serait approprié pour décrire la
distribution des données de l'échantillon serait rejetée.

Notons qu'on peut aussi estimer les paramètres de chacun des cas
particuliers directement par la méthode des moments généralisée en
utilisant des conditions de moment basées sur la moyenne $m_1$ et la
variance $m_2$ de la forme \eqref{eq:condmom}. Par contre, dans ces
cas, on devra utiliser des tests d'adéquation non paramétriques afin
de décider laquelle représente mieux la distribution de l'échantillon
de données. Ces tests seront présentés au chapitre suivant, à la
section \ref{sec:testnonparam}.

\section{Méthode de l'équation d'estimation optimale}
\label{sec:CrowderGAL}

On utilise maintenant la méthode de l'équation d'estimation optimale
afin d’estimer les paramètres de la distribution de Laplace
asymétrique généralisée. On considère une équation d'estimation de la
forme quadratique \eqref{eq:generalquad}, basée sur les conditions de
moments \eqref{eq:condmomGAL} utilisées à la section précédente.

On évalue d'abord les expressions des dérivées premières de la moyenne
et de l'écart-type par rapport au vecteur de paramètres
\eqref{eq:derivmom}:
\begin{subequations}\label{eq:derivmomGAL}
  \begin{align}
    \mu^{\prime}\left( \theta \right) &= \frac{d}{d\theta}\left(\theta+\tau\,\mu\right)  \nonumber\\
    &= \left[ \begin{array}{c} 1\\
    0\\
    \tau\\
    \mu\end{array} \right]^T \label{eq:derivmom1GAL}\\
    \sigma^{\prime}\left( \theta \right) &= \frac{d}{d\theta}\sqrt{\tau\sigma^2+\tau\mu^2} \nonumber\\
    &= \frac{1}{\sqrt {\tau\,{\sigma}^{2}+\tau\,{\mu}^{2}}}
    \left[ \begin{array}{c}
        0\\
        {\tau\,\sigma}\\
        {\tau\,\mu}\\
        \frac{{\sigma}^{2}+{\mu}^{2}}{2}
      \end{array} \right]^T.\label{eq:derivmom2GAL}
  \end{align}
\end{subequations}

On peut alors obtenir une forme analytique en utilisant les
expressions déterminées précédemment pour $\mathbf{a}(\theta)$ et
$\mathbf{b}(\theta)$ \eqref{eq:coefficientscrowder}, les moments de la
distribution \eqref{eq:momentsGAL} ainsi que les coefficients
d'asymétrie et d'aplatissement \eqref{eq:moments56GAL}. Étant donné la
longueur des expressions obtenues, elles ne seront pas présentées dans
ce texte. Cependant, on peut facilement les évaluer à l'aide d'un
logiciel de calcul symbolique. On combine ensuite celles-ci pour
obtenir l'équation d'estimation optimale $g\left(\theta;\mathbf{y}_T
\right)$ de forme quadratique.

On obtient cependant des expressions plus faciles à manipuler pour
$\mathbf{a}(\theta)$ et $\mathbf{b}(\theta)$ en utilisant la version
modifiée du vecteur de pondération optimal
\eqref{eq:vecteurcrowdermod}:
\begin{subequations}
  \label{eq:vecteurcrowdermodGAL}
  \begin{align}
    \mathbf{a}_{mod}^{*}(\theta;\mathbf{y}_T) &= \begin{bmatrix}
      \frac{-\hat\gamma_2\left(\mathbf{y}_T\right)-2}{\left( {\sigma}^{2}+{\mu}^{2}\right) \,\tau\,\hat\gamma_3\left(\mathbf{y}_T\right) } \\
      \frac{2\,\sigma\,\hat\gamma_1\left(\mathbf{y}_T\right)}{{\left( {\sigma}^{2}+{\mu}^{2}\right) }^{3/2}\,\sqrt{\tau}\,\hat\gamma_3\left(\mathbf{y}_T\right) } \\
      \frac{\frac{2\,\mu\,\sqrt{\tau}\,\hat\gamma_1\left(\mathbf{y}_T\right)}{\sqrt{{\sigma}^{2}+{\mu}^{2}}}+\tau\,\left( -\hat\gamma_2\left(\mathbf{y}_T\right)-2\right) }{\left( {\sigma}^{2}+{\mu}^{2}\right) \,\tau\,\hat\gamma_3\left(\mathbf{y}_T\right) } \\
      \frac{\frac{\sqrt{{\sigma}^{2}+{\mu}^{2}}\,\hat\gamma_1\left(\mathbf{y}_T\right)}{\sqrt{\tau}}+\mu\,\left(
          -\hat\gamma_2\left(\mathbf{y}_T\right)-2\right) }{\left(
          {\sigma}^{2}+{\mu}^{2}\right)
        \,\tau\,\hat\gamma_3\left(\mathbf{y}_T\right) }
    \end{bmatrix} \\
    \mathbf{b}_{mod}^{*}(\theta;\mathbf{y}_T) &= \begin{bmatrix}
      \frac{\hat\gamma_1\left(\mathbf{y}_T\right)}{{\left( {\sigma}^{2}+{\mu}^{2}\right) }^{3/2}\,{\tau}^{3/2}\,\hat\gamma_3\left(\mathbf{y}_T\right) } \\
      -\frac{2\,\sigma}{{\left( {\sigma}^{2}+{\mu}^{2}\right) }^{2}\,\tau\,\hat\gamma_3\left(\mathbf{y}_T\right) } \\
      \frac{\tau\,\hat\gamma_1\left(\mathbf{y}_T\right)-\frac{2\,\mu\,\sqrt{\tau}}{\sqrt{{\sigma}^{2}+{\mu}^{2}}}}{{\left( {\sigma}^{2}+{\mu}^{2}\right) }^{3/2}\,{\tau}^{3/2}\,\hat\gamma_3\left(\mathbf{y}_T\right) } \\
      \frac{\mu\,\hat\gamma_1\left(\mathbf{y}_T\right)-\frac{\sqrt{{\sigma}^{2}+{\mu}^{2}}}{\sqrt{\tau}}}{{\left(
            {\sigma}^{2}+{\mu}^{2}\right)
        }^{3/2}\,{\tau}^{3/2}\,\hat\gamma_3\left(\mathbf{y}_T\right) }
    \end{bmatrix}.
  \end{align}
\end{subequations}

On insère ensuite ces dernières dans la forme générale
\eqref{eq:generalquad} pour obtenir l'équation d'estimation optimale
modifiée $g_{mod}^{*}\left(\theta;Y \right)$:
\begin{align}
  \label{eq:generalquadmod}
  g_{mod}^{*}\left(\theta;Y \right) = \sum_{t=1}^n \left[
    \mathbf{a}_{mod}^{*}(\theta;\mathbf{y}_T)(y_t-\mu\left(\theta\right)) +
    \mathbf{b}_{mod}^{*}(\theta;\mathbf{y}_T)\left((y_t-\mu\left(\theta\right))^2-\sigma^2\left(\theta\right)
    \right)\right].
\end{align}

On note qu'afin d'éviter des irrégularités numériques, on suggère,
dans les deux cas, de faire l'estimation sur des données centrées et
réduites:
\begin{align}
  X_t &=
  \frac{Y_t-\hat{m_1}}{\sqrt{\hat{m_2}}}. \label{eq:defcentrereduite}
\end{align}

Comme précédemment, on pourra utiliser la propriété d'invariance
\eqref{eq:GALscaletrans} de la paramétrisation en $\kappa$ afin de
retrouver les paramètres de la distribution de la variable aléatoire
$Y_t$ une fois l'estimation effectuée:
\begin{align}
  Y_t &= \hat{\sigma}_{t} X_t + \hat{\mu}_{t} \sim
  GAL(\hat{\sigma}_{t} \theta +
  \hat{\mu}_{t},\hat{\sigma}_{t}\sigma,\kappa,\tau). \label{eq:paramnonreduit}
\end{align}

Afin d'obtenir le vecteur de paramètres estimés $\hat\theta$, on
minimise la fonction objectif
$\Lambda\left(\theta;\mathbf{y}_T\right)$
\eqref{eq:eqnobjectifEE}. Une fois les paramètres obtenus, on peut
ensuite évaluer la variance-covariance de ces estimateurs, à partir du
résultat \eqref{eq:Moptimalestimetheta}.

%%% Local Variables: 
%%% mode: latex
%%% TeX-master: "gabarit-maitrise"
%%% End: 
