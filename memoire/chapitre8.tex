\chapter{Évaluation d'options}
\label{chap:options}
Un des principaux intérêts de connaître la distribution des rendements
d'un actif est de pouvoir évaluer la valeur de différents produits
dérivés.

\section{Définitions}
\label{sec:options}
En se référant à \cite{bingham2004risk}, on définit:

\begin{itemize}
\item Un \textbf{produit dérivé} est un contrat financier dont la
  valeur à la date d'échéance $T$ est déterminée par le prix de
  l'actif sous-jacent au temps $T$ ou par celles prises au cours de
  l'intervalle $\left[0,T \right]$.
\item Une \textbf{option} est un instrument financier qui donne le
  droit (et non l'obligation) d'effectuer une transaction avant ou à
  une date et pour un prix spécifiés.
\item Une \textbf{option d'achat (de vente) européenne} donne le
  \textbf{droit} d'acheter (de vendre) un actif au \textbf{prix
    d'exercice} $K$ au temps $T$.
\item Lorsque sa valeur actuelle est, par rapport au prix
  d'exercice:
  \begin{itemize}
  \item supérieure $(S(t)>K)$, l'option d'achat est dite \textbf{dans le
      cours};
  \item égale $(S(t)=K)$, l'option d'achat est dite \textbf{au cours};
  \item inférieure $(S(t)<K)$, l'option d'achat est dite \textbf{hors du
      cours}.
  \end{itemize}
\end{itemize}

Le prix à l'échéance d'une option d'achat européenne $C(t)$ est défini
comme suit:
\begin{align}
  \label{eq:callpayoff}
  C(T) = \left\{ \begin{array}{cc} 
      S(T)-K & ,S(T) > K \\
      0 & S(T) \leq K.
    \end{array} \right.
\end{align}

La valeur du prix à l'échéance est appelée une \textbf{créance
  éventuelle}. L'évaluation d'un produit dérivé équivaut à calculer la
valeur espérée actualisée de la réclamation contingente définie par le
contrat. Pour l'option d'achat européenne, on obtient la formule
suivante:
\begin{align}
  C(t) &= B(t,T) E \left[C(T) \right] \nonumber\\
  &= B(t,T) \int_{K}^{\infty} (S(T)-K)
  d\hat{F}_{S(T)} \label{eq:reclamationcall}.
\end{align}
$B(t,T)$ est la valeur au temps $t$ d'une obligation zéro-coupon au taux sans
risque $r$ d'échéance $T$. $\hat{F}_t(S(T))$ est la fonction de
répartition de la distribution neutre au risque de $S(T)$.

Les déterminants de la valeur d'une option sont:
\begin{itemize}
\item le prix courant de l'actif $S(t)$
\item le prix d'exercice $K$
\item la volatilité de l'actif $\sigma$
\item le temps d'ici l'échéance $T-t$
\item le taux d'intérêt sans risque $r$.
\end{itemize}

On peut aussi récrire \eqref{eq:reclamationcall} avec le logarithme du
prix $k=\ln{K}$. On utilise donc $s(t)=\ln{S(T)}$, dont la fonction de
répartition est définie par ${F}_{s(T)}(y) = Pr\left[\ln{S(T)} < y
\right]$:
\begin{align}
  C(t) = B(t,T) \int_{k}^{\infty} (e^{s(T)}-e^{k}) d{F}_{s(T)}. \label{eq:reclamationcallintlog}
\end{align}

Enfin, une relation fondamentale, appelée parité vente-achat, permet,
dans un scénario sans arbitrage, de calculer le prix d'une option de
vente $P(t)$ (d'achat $C(t)$) en connaissant la valeur:
\begin{itemize}
\item de l'autre lorsqu'elles sont de mêmes échéance et prix
  d'exercice
\item de l'obligation $B(t,T)$
\item initiale du titre sous-jacent $S(t)$ commun aux deux options.
\end{itemize}

\begin{align}
  \label{eq:pariteputcall}
  P(t)+S(t) = C(t)+ B(t,T) K.
\end{align}

Cette relation permettra d'évaluer les deux types d'option à l'aide
d'un seul calcul.

\subsection{Équation martingale}
\label{sec:equationmartingale}
L'évaluation d'options se fait traditionnellement dans un univers ou
un espace de probabilités neutre au risque, noté $\mathbb{Q}$,
c'est-à-dire un point de vue selon lequel les investisseurs n'exigent
pas une prime de risque. Cette approche a été introduite par
\cite{black1973pricing}. Dans cette perspective, les rendements
espérés futurs sont escomptés au taux sans risque. La distribution des
rendements cumulés \eqref{eq:rendementcumL} répond à la propriété
martingale, selon laquelle la valeur actuelle d'un titre financier
reflète l'ensemble de l'information connue sur ce dernier. Cette
propriété permet de fixer un seul prix pour les
options. \textbf{L'équation martingale} établit l'égalité entre la
valeur espérée du titre au temps $t$ et celle d'un investissement de
même valeur dans un compte en banque créditant le taux sans risque:
\begin{align}
  \label{eq:equationmartingale}
  E\left[\exp(L_t) \right] = e^{rt}.
\end{align}

\subsection{Paramètres neutres au risque}
\label{sec:GALrn}

Afin de pouvoir utiliser les résultats de l'estimation paramétrique
des chapitres précédents pour évaluer le prix de produits dérivés, on
doit tout d'abord identifier les paramètres neutres au risque de la
distribution de Laplace asymétrique généralisée associant le
rendement cumulé $L_t$ au taux d'intérêt sans risque $r$. Pour ce
faire, on utilise l'équation martingale \eqref{eq:equationmartingale}
ainsi que la fonction génératrice des moments \eqref{eq:fgmGAL}. On
obtient alors une expression pour le paramètre de dérive neutre au
risque $\theta_{RN}$:
\begin{align}
  e^{rt} &= E\left[\exp(L_t) \right]
  = E\left[\exp(R_1+\ldots+R_t) \right] \nonumber\\
  &= M_{R_1+\ldots+R_t}(1)
  = (M_{R}(1))^t \nonumber\\
  &= \left(\frac{e^{\theta_{RN}}}{(1-\mu-\sigma^2/2)^{\tau}} \right)^t \nonumber\\
  r &= \ln \left(\frac{e^{\theta_{RN}}}{(1-\mu-\sigma^2/2)^{\tau}} \right) \nonumber\\
  &= \theta_{RN} - \tau\ln(1-\mu-\frac{\sigma^2}{2}) \nonumber\\
  \theta_{RN} &= r +
  \tau\ln(1-\mu-\frac{\sigma^2}{2}). \label{eq:martingaleGAL}
\end{align}

On obtient ainsi le paramètre de correction de la dérive $\omega$ à
partir de l'équation martingale \eqref{eq:martingaleGAL}:
\begin{align}
  \label{eq:omegaGAL}
  \omega = \tau\ln(1-\mu-\frac{\sigma^2}{2}).
\end{align}

Cette égalité impose une contrainte aux paramètres $\mu$ et $\sigma$
qui doit respecter la condition
\begin{align}
  \label{eq:inegaliteparamrn}
  \mu+\frac{\sigma^2}{2} < 1.
\end{align}

On obtient ensuite la fonction caractéristique de la mesure neutre au
risque $\mathbb{Q}$ pour $L_T$ en remplaçant $\theta$ par
$\theta_{RN}$:
\begin{align}
  \label{eq:fncaractGALrn}
  \phi(\xi,L_T) = \frac{\exp\left(i\xi*\left(\ln\left(S_{t}\right)+\left(T-t\right)\left(r+\omega\right)\right)\right)}{\left(1-i\mu\,\xi+\left(\sigma^2\xi^2\right)/2\right)^{\tau\left(T-t\right)}}.
\end{align}

La distribution neutre au risque des rendements est donc $R_t \sim
GAL(r+\tau\ln(1-\mu-\sigma^2/2),\sigma,\mu,\tau)$.

\section{Aperçu du modèle de Black-Scholes}
\label{sec:blackscholes}

Étant donné l'importance du modèle de Black-Scholes en finance, on se
doit de le présenter comme outil de référence lorsque l'on veut le
remplacer. Ce modèle est basé sur les quatre hypothèses suivantes:
\begin{enumerate}
\item Les rendements ont une distribution normale de moyenne $\mu$ et
  variance $\sigma^2$, ou, de manière équivalente, le processus de
  prix $\lbrace S(t) \rbrace$ est un mouvement brownien géométrique de
  dérive $\mu$ et de volatilité $\sigma$.
\item Une obligation zéro-coupon au taux d'intérêt sans risque $r$
  existe sur le marché.
\item Le marché est complet et sans friction. Une position prise sur
  ce marché peut être couverte de manière continue.
\item La vente à découvert est autorisée, de plus tous les actifs sont
  infiniment divisibles.
\end{enumerate}

En utilisant l'équation martingale \eqref{eq:equationmartingale}, on
obtient la distribution neutre au risque des rendements, qui est
normale avec une moyenne $r-\frac{\sigma^2}{2}$ et une variance
$\sigma^2$.

À partir de cette information, on peut dériver la formule de
Black-Scholes pour le prix d'une option d'achat ou de vente
\eqref{eq:pariteputcall} européenne:
\begin{align}
  \label{eq:blackscholes}
  C(S(t),K,T) &= \Phi(\Delta_1) ~ S(t) - \Phi(\Delta_2) ~ K B(t,T) \\
  \Delta_1 &= \frac{\ln\left(\frac{S(t)}{K}\right) + \left(r + \frac{\sigma^2}{2}\right)(T - t)}{\sigma\sqrt{T - t}} \nonumber\\
  \Delta_2 &= \Delta_1 - \sigma\sqrt{T - t} \nonumber\\
  P(S(t),K,T) &= \Phi(-\delta_2)K B(t,T) -
  \Phi(-\delta_1)S(t). \label{eq:putblackscholes}
\end{align}

L'avantage de ce modèle est que toutes les données requises pour
évaluer le prix d'options sur des actifs cotés sur les marchés
boursiers, à l'exception de la volatilité, sont accessibles auprès de
fournisseurs d'informations financières. Comme les options sont
négociées sur les marchés financiers, on a aussi accès à leur prix, ce
qui permet d'extrapoler la valeur de la volatilité implicite. C'est
pour cette raison que le modèle, bien qu'il soit basé sur des
hypothèses très restrictives, est toujours utilisé comme point de
référence.

\section{Méthodes d'évaluation pour options européennes}
\label{sec:methodesoptions}

Le modèle de Black-Scholes est un des seuls qui présentent une forme
analytique simple pour le prix des options. Bien qu'on puisse en
dériver une pour plusieurs modèles en utilisant les équations
différentielles stochastiques et le calcul d'Îto, les résultats sont
souvent difficiles à utiliser. On préfèrera alors utiliser des
méthodes numériques pour calculer le prix des options.

\subsection{Méthode de Heston}
\label{sec:heston1993}

L'approche utilisée par \cite{heston1993closed} est de calculer
directement l'espérance de la créance éventuelle associée à l'option
de vente européenne $P(S(t),K,T-t)$ sous la mesure neutre au risque
$\mathbb{Q}$:
\begin{align}
  P(S(t),K,T) &= B(t,T) \int_{0}^{\infty} \max\left(K-S(T),0\right)\cdot d{F}_{S(T)}  \label{eq:PutHeston93-1}\\
  &= B(t,T)K \int_{0}^{K} dF_{S(T)} - B(t,T)\int_{0}^{K} S(T)\cdot d{F}_{S(T)} \nonumber\\
  &= B(t,T)K F_{S(t)}(K) - S(t) G_{S(t)}(K). \label{eq:PutHeston93}
\end{align}

On définit la fonction de répartition $F_{S(t)}(K)$ de la variable
aléatoire $S(t) \leq K$ sous la mesure $\mathbb{Q}$. On définit aussi
la fonction de répartition $G_{S(t)}(K)$ sous une autre mesure
équivalant à $\mathbb{Q}$. Soit $\phi_F(\xi)$ la fonction
caractéristique de $S(T)$ sous $\mathbb{Q}$, celle de cette mesure est
alors définie comme suit:
\begin{align}
  \label{eq:mesureGHeston}
  \phi_G(\xi) = \frac{\phi_F(\xi-i)}{\phi_F(-i)}.
\end{align}
C'est une transformée d'Esscher de paramètre $h=1$ de la mesure
$\mathbb{Q}$. On remarquera la forme de \eqref{eq:PutHeston93} qui est
très similaire à la formule du prix de l'option de vente du modèle de
Black-Scholes \eqref{eq:putblackscholes}.

Les fonctions de répartition $F_{S(t)}(K)$ et $G_{S(t)}(K)$
peuvent être évaluées à partir de \eqref{eq:approxinvfncaract} ou de
la méthode du point de selle.

\cite{carr1999option} notent qu'on ne peut pas inverser
l'approximation \eqref{eq:approxinvfncaract} en utilisant la méthode
de la transformée de Fourier rapide. Cependant, l'approche de la
méthode du point de selle n'a pas été considérée, bien qu'elle puisse
fournir une solution analytique dans plusieurs situations, en
particulier pour la distribution de Laplace asymétrique généralisée
\eqref{eq:saddlepointGAL}.

\subsection{Méthode de Carr et Madan}
\label{sec:carrmadanfftoptions}

La méthode de Heston nécessite l'évaluation de deux probabilités, donc
deux inversions de fonctions
caractéristiques. \cite{madan1998variance} ont développé une méthode
qui ne nécessite qu'une seule inversion, décrite par
\cite{epps2007pricing}. Cette méthode nécessite cependant la sélection
d'un paramètre d'amortissement $\theta_{D}$ par tâtonnement afin
d'obtenir de bons résultats. Elle utilise l'expression de la créance
éventuelle \emph{amortie} \footnote{traduction de l'anglais damped} de
l'option d'achat, construite à partir de l'expression
\eqref{eq:reclamationcallintlog}:
\begin{align}
  \label{eq:callamortiCarr}
  C_{\theta_D}(S(t),k,T) &= e^{\theta_{D}k} C(S(t),k,T) \\
  &= e^{\theta_{D}k} B(t,T) \int_{K}^{\infty} (e^{s(T)}-e^{k})
  d{F}_{s(T)}.
\end{align}

On applique la transformée de Fourier:
\begin{align}
  \label{eq:fouriercallamortiCarr}
  \mathcal{F}C_{\theta_D}(\nu) = B(t,T) \frac{\phi\left( \nu - i(1+\theta_D) \ \right)}{(\theta_D+i\nu)(1+\theta_D+i\nu)}.
\end{align}

Ce résultat est inversé et \emph{amplifié} \footnote{traduction de
  l'anglais amplified} pour retrouver le prix de l'option:
\begin{align}
  C(S(t),k,T) &= e^{-\theta_{D}k} \mathcal{F}^{-1}\left(\mathcal{F}C_{\theta_D}(\nu) \right) \nonumber\\
  &= \frac{e^{-\theta_{D}k}}{2\pi} \int_{-\infty}^{\infty} e^{-i\nu k}
  (\mathcal{F}C_{\theta_D})(\nu)\cdot
  d\nu \nonumber\\
  &= \frac{e^{-\theta_{D}k}}{\pi} \int_{0}^{\infty} e^{-i\nu k}
  (\mathcal{F}C_{\theta_D})(\nu)\cdot d\nu \qquad \mbox{(par
    symétrie)}. \label{eq:prixoptionCarr}
\end{align}

On peut utiliser la méthode de la transformée de Fourier rapide
(section \ref{sec:methodeFFT}) pour retrouver les prix d'options en se
référant à \cite{carr1999option}.

On doit donc discrétiser \eqref{eq:prixoptionCarr} et effectuer des
changements de variables pour obtenir la forme \eqref{eq:sommefft}. On
fixe le nombre de points $N$ (idéalement une puissance de deux) et
$\eta$ le pas de discrétisation. On définit l'incrément $\lambda$ pour le
vecteur $k_u$, $u=1,\ldots,N$ des logarithmes des prix d'exercice:
\begin{align}
  \lambda = \frac{2\pi}{N\eta}.
\end{align}

La valeur maximale $b$ de ce prix d'exercice sera:
\begin{align}
  b=\frac{N\lambda}{2}.
\end{align}

Enfin, on obtient la suite des points d'évaluation de la fonction
caractéristique $\nu_j = \eta (j-1)$. L'ensemble de ces substitutions
permet d'obtenir une expression qui est de la forme requise pour
appliquer la méthode de la transformée de Fourier rapide.
\begin{align}
  C(S(t),k_u,T) &\approx \frac{e^{-\theta_D\,k}}{\pi} \sum_{j=1}^N e^{-i\nu_j\,k} (\mathcal{F}C_{\theta_D})(\nu_j) \nonumber\\
  &\approx \frac{e^{-\theta_D\,k}}{\pi} \sum_{j=1}^N
  e^{-i\lambda\eta(j-1)(u-1)} e^{ib\nu_j}
  (\mathcal{F}C_{\theta_D})(\nu_j) \eta. \label{eq:prixoptionFFT}
\end{align}

C'est essentiellement une intégration par la méthode du trapèze. On
suggère d'implémenter la correction de Simpson, qui permet une
intégration plus précise tout en conservant le même nombre de points
de discrétisation. On obtiendra alors la formule suivante:
\begin{align}
  C(S(t),k_u,T) &\approx \frac{e^{-\theta_D\,k}}{\pi} \sum_{j=1}^N
  e^{-i\lambda\eta(j-1)(u-1)} e^{ib\nu_j}
  (\mathcal{F}C_{\theta_D})(\nu_j) \frac{\eta}{3} \left(3 + (-1)^j +
    \delta_{j-1}\right). \label{eq:prixoptionFFTsimpson}
\end{align}

On peut par la suite calculer un prix d'option pour chaque prix
d'exercice en utilisant une méthode d'interpolation (splines cubiques
par exemple).

\subsection{Prix d'exercice hors du cours}
\label{sec:outofmoneyfft}

L'intégration numérique de \eqref{eq:prixoptionCarr} pose problème
quand l'échéance $T-t$ est petite, ou encore le prix d'exercice $K$
est hors du cours ($k > \ln(S(t))$). Dans ce contexte particulier,
\cite{carr1999option} développent une formule alternative à l'équation
\eqref{eq:prixoptionCarr} pour évaluer le prix de l'option d'achat :
\begin{align}
  C(S(t),k,T) &= \frac{1}{\pi\sinh(\theta^{*}_D k)} \int_{0}^{\infty}
  e^{-i\nu k} (\mathcal{F}C^{*}_{\theta^{*}_D})(\nu)\cdot
  d\nu. \label{eq:prixoptionCarrOOM}
\end{align}
La transformée de Fourier de l'expression du prix de l'option d'achat
européenne est exprimée sous la forme
\begin{align}
  \mathcal{F}C^{*}_{\theta^{*}_D} &= \frac{\zeta_T(\nu-i\theta^{*}_D) - \zeta_T(\nu+i\theta^{*}_D)}{2}\\
  \zeta_T(\nu) &= e^{-r(T-t)}\bigg[\frac{1}{1+i\nu} -
  \frac{e^{r(T-t)}}{iv} - \frac{\phi(\nu-i)}{\nu(\nu-i)} \bigg].
  \nonumber
\end{align}

On note que ce paramètre d'amortissement $\theta^{*}_D$ peut être
différent du paramètre $\theta_D$ qui a été utilisé précédemment.

L'expression à utiliser pour la méthode de la transformée de Fourier
rapide devient
\begin{align}
  C^{*}(S(t),k_u,T) &\approx \frac{1}{\pi\sinh(\theta^{*}_D k)}
  \sum_{j=1}^N e^{-i\lambda\eta(j-1)(u-1)} e^{ib\nu_j}
  (\mathcal{F}C^{*}_{\theta^{*}_D})(\nu_j) \frac{\eta}{3} \left(3 +
    (-1)^j + \delta_{j-1}\right). \label{eq:prixoptionFFTsimpsonOOM}
\end{align}
Les paramètres $N ,\eta, \lambda$ et $\nu_j$ prennent les valeurs
utilisées pour évaluer le prix dans la situation où le titre est dans
le cours \eqref{eq:prixoptionFFTsimpson}.

\subsection{Critique de la méthode de Carr-Madan}
\label{sec:critiquecarrmadanfft}

Un paramètre d'amortissement $\theta_{D}$ inférieur à la valeur
optimale (lorsque comparé avec la méthode de Heston ou d’Epps) aura
tendance à surestimer le prix des options d'achat
européennes. \cite{itkin2005pricing} démontre que pour certaines
régions de l'espace des paramètres, l'intégrale
\eqref{eq:prixoptionCarr} a un comportement très irrégulier. De plus,
il décrit plusieurs restrictions pour $\theta_{D}$ qui, dans plusieurs
cas, ne permettent pas d'estimation convergente.

\subsection{Méthode d’Epps}
\label{sec:epps2007}

\cite{epps2007pricing} propose une méthode qui, contrairement à celle
de Carr et Madan, ne nécessite pas de paramètre d'amortissement et qui
ne requiert aussi qu'une seule inversion de la fonction
caractéristique. Pour ce faire, on se base sur l'expression de
l'option de vente développée précédemment \eqref{eq:PutHeston93-1}, qui
est ensuite exprimée sous la forme du logarithme, à la manière de
\eqref{eq:reclamationcallintlog}.  Après l'utilisation de la formule
d'inversion \eqref{eq:gilpelaez2}, on obtient le résultat suivant:
\begin{align}
  \label{eq:putepps-1}
  E\left[max(K-S_T;0) \right] &= \int_{-\infty}^{k} {F}_{S(T)}(s)e^s\cdot\,ds \nonumber\\
  &= \frac{K}{2}-\frac{1}{2}\int_{-\infty}^{k} \lim_{c\to\infty}
  \underbrace{\int_{-c}^{c} \frac{e^{-i\nu s}}{\pi
      i\nu}\phi(\nu)d\nu}_{a(c,s)}e^sds.
\end{align}

La formule d'inversion implique que lorsque $|\lim_{c\to\infty}
a(c,s)| = |1-2{F}_{S(t)}(s)|\leq 1$, pour toute valeur $\epsilon>0$,
$c_{\epsilon}$ existe telle que l'on a le résultat suivant:
\begin{align}
  \label{eq:putepps-resultat1}
  |sup_s\left[1-2{F}_{S(t)}(s)-a(c_{\epsilon},s) \right]|<\epsilon.
\end{align}

Dans cette situation, $|a(c,s)e^s|\leq e^s(1+\epsilon)$ lorsque la
constante $c$ est suffisamment grande. Comme la fonction $e^s$ est
intégrable sur le support $(-\infty,k)$, le théorème de convergence
dominée de Lebesgue (section \ref{sec:theor-de-conv}) implique que
l'on peut poser l'égalité suivante:
\begin{align}
  \int_{-\infty}^{k} \lim_{c\to\infty} a(c,s)e^sds &= \lim_{c\to\infty}\int_{-\infty}^{k}  a(c,s)e^sds \nonumber\\
  &= \lim_{c\to\infty} \int_{-\infty}^{k}\int_{-c}^{c}
  \frac{e^{(1-i\nu) s}}{\pi i\nu}\phi(\nu)d\nu ds. \nonumber
\end{align}

De plus, comme la limite de la double intégrale précédente est égale à
$K-2E\left[max(K-S_T;0) \right]$ qui appartient à l'intervalle
$\left[-K,K\right]$, le théorème de Fubini (section
\ref{sec:theoreme-de-fubini}) permet d'inverser l'ordre d'intégration:
\begin{align}
  \lim_{c\to\infty} \int_{-\infty}^{k}\int_{-c}^{c} \frac{e^{-i\nu
      s}}{\pi i\nu}\phi(\nu)d\nu ds &= \lim_{c\to\infty} \int_{-c}^{c} \int_{-\infty}^{k}e^{(1-i\nu)s} ds \frac{\phi(\nu)}{\pi i \nu} d\nu \nonumber\\
  &= \frac{K}{\pi} \lim_{c\to\infty} \int_{-c}^{c}
  \frac{K^{-i\nu}}{i\nu+\nu^2}\phi(\nu)d\nu. \label{eq:fubini-integrale-a-EPPS}
\end{align}

En remplaçant le résultat \eqref{eq:fubini-integrale-a-EPPS} dans
l'équation de départ \eqref{eq:putepps-1}, on obtient ainsi une
expression particulièrement simple pour le prix de l'option de vente:
\begin{align}
  P(S(t),K,T) &= B(t,T)K\left[\frac{1}{2}-\frac{1}{2\pi} \int_{-c}^c
    K^{-i\nu} \frac{\phi(\nu)}{\nu(i+\nu)}
    d\nu\right]. \label{eq:PutEpps8.37}
\end{align}

Cependant, puisque la forme ne se prête pas à l'utilisation de
l'algorithme de la transformée de Fourier rapide, on a recours à une
procédure d'intégration numérique.

\section{Particularités}
\label{sec:monnaiescontrats}

\subsection{Option sur actions avec dividendes}
\label{sec:dividentoptions}

Lorsqu'une option a, pour titre sous-jacent, une action qui verse des
dividendes, on doit en tenir compte dans l'évaluation de son prix. Si
l'on considère que la valeur au marché de l'action $S^{*}(t)$ a été
évaluée avec la méthode de l'actualisation des flux financiers futurs,
on doit tenir compte de la valeur actualisée des dividendes qui seront
versés avant l'échéance de l'option et la soustraire de ce prix.

Parfois, les dividendes ne sont pas fixes, mais proportionnels à la
valeur de l'action à la date ex-dividende avec un taux $\delta$. Soit
$n(T)$ le nombre de dates ex-dividende dans l'intervalle $\left] t,T
\right]$, le prix initial $S(t)$ considéré pour le calcul de la valeur
de l'option est alors défini comme suit:
\begin{align}
  \label{eq:prixinitialdividende}
  S(t) = S^{*}(t) (1-\delta)^{n(T)}.
\end{align}

Lorsque l'on considère un indice composé de plusieurs titres, on peut
prendre un dividende versé de manière continue à un taux $q$. Dans ce
cas, le prix initial $S(t)$ sera
\begin{align}
  \label{eq:pricinitialdivcontinu}
  S(t) = S^{*}(t) e^{-q(T-t)}.
\end{align}

\subsection{Options sur contrats à terme et taux de change}
\label{sec:futureoptions}

\cite{black1976pricing} (p.177) démontre que le prix d'une option sur
un contrat à terme a le même prix que sur une action dont le taux de
dividende équivaut au taux sans risque. On peut donc utiliser les
résultats \eqref{eq:prixinitialdividende} et
\eqref{eq:pricinitialdivcontinu} en posant $q=r$.

De même, on peut évaluer le prix d'une option sur une monnaie
étrangère en considérant le taux sans risque étranger $r_f$ comme un
taux de dividende $q=r_f$. Selon la Banque des Règlements
Internationaux, c'est le type d'options le plus transigé sur les
marchés non réglementés en date de 2005, même s'il est moins étudié
que les autres.







%%% Local Variables: 
%%% mode: latex
%%% TeX-master: "gabarit-maitrise"
%%% End: 
