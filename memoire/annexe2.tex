\chapter{Éléments de statistique mathématique}

\section{Loi faible des grands nombres}
\label{sec:loifaible}

La \textbf{loi faible des grands nombres} est un résultat important en
probabilité, car il permet de définir la notion d'estimateur
convergent. Soit une suite de variables aléatoires indépendantes et
identiquement distribuées $\left\{X_T \right\}_{T=1}^{\infty}$ ayant
une espérance $E\left[ X \right]$ et une variance $V\left[ X \right]$
finies. Selon la loi faible des grands nombres, pour tout nombre réel
strictement positif $\varepsilon$, la probabilité que la différence
entre la moyenne empirique $Y_T=\frac{X_1+X_2+\cdots+X_T}{T}$ et
l'espérance $E\left[ X \right]$ soit supérieure à la valeur
$\varepsilon$ tend vers 0 lorsque $T$ tend vers l'infini.
\begin{align}
  \label{loifaible}
  \lim_{T \to +\infty} \mathbb{P}\left(\left|Y_T - E\left[ X
      \right]\right| \geq \varepsilon\right) = 0 ,\quad
  \forall\varepsilon>0
\end{align}

On dit alors que la suite d'estimateurs
$\left\{Y_T\right\}_{T=1}^{\infty}$ converge en probabilité vers
l'espérance $E\left[ X \right]$. L'estimateur de l'espérance $Y_T$ est
alors \textbf{convergent}.

\section{Théorème central limite}
\label{sec:theor-centr-limite}

Le \textbf{théorème central limite} est un résultat fondamental en
probabilité qui énonce le rôle de la distribution normale. Il démontre
que toute somme de variables aléatoires indépendantes et identiquement
distribuées suit approximativement une loi normale. Ce résultat permet
entre autres d'identifier la distribution limite d'un estimateur
convergent.

\subsection{Cas univarié}
\label{sec:cas-univarie}

Soit une suite d'observations $X_1, \ldots, X_T$ d'un échantillon
aléatoire d'une distribution de moyenne $\mu$ et de variance
$\sigma^2$:
\begin{align}
  \label{eq:TCL}
  Y_T &= \frac{1}{\sqrt{T}\sigma} \left(\sum_{t=1}^T X_t - T\mu\right) \nonumber\\
  &= \frac{\sqrt{T}}{\sigma}\left(\overline{X}_T-\mu\right).
\end{align}
Alors, cette variable aléatoire converge en distribution vers une
variable aléatoire normale centrée réduite:
\begin{align}
  \label{eq:TCL2}
  Y_T \stackrel{L}{\rightarrow} \mathcal{N}(0,1).
\end{align}

\subsection{Cas multivarié}
\label{sec:cas-multivarie}

On peut aussi généraliser ce théorème pour des observations
multivariées. On considère alors une série d'observations
multivariées $\mathbf{X_1}, \ldots, \mathbf{X_T}$ où
\begin{align}
  \label{eq:defmultiX}
  \mathbf{X_t}=\begin{bmatrix} X_{t(1)} \\ \vdots \\
    X_{t(k)} \end{bmatrix}, \quad t=1,\ldots,T.
\end{align}

On définit maintenant la variable aléatoire correspondante:
\begin{align}
  \label{eq:TCLmulti1}
  \mathbf{Y}_T &= \frac{1}{T}\begin{bmatrix} \sum_{t=1}^{T} \left [
      X_{t(1)} \right ] \\ \vdots \\ \sum_{t=1}^{T} \left [ X_{t(k)}
    \right ] \end{bmatrix}.
\end{align}

Dans cette situation, la variable aléatoire converge en distribution
vers une variable aléatoire de distribution normale multivariée
centrée de matrice de variance-covariance $\mathbf{\Sigma}$:
\begin{align}
  \label{eq:TCLmulti2}
  \sqrt{T}\left(\mathbf{Y}_T - \boldsymbol{\mu}\right)\ \stackrel{L}{\rightarrow}\
  \mathcal{N}_k(0,\mathbf{\Sigma})
\end{align}

où
\begin{align*}
  \mathbf{\Sigma} &= \begin{bmatrix}
    \omega_{(1,1)} &\cdots& \omega_{(1,k)} \\
    \vdots & \ddots & \vdots \\
    \omega_{(k,1)} &\cdots& \omega_{(k,k)} \\
  \end{bmatrix} ,\quad \mbox{avec } \omega_{(j,k)} = \begin{cases}
    Var\left[X_{1(j)}\right] , &(j=k) \\
    Cov\left[X_{1(j)},X_{1(k)}\right] , &(j \neq k).
  \end{cases}
\end{align*}

\section{Méthode delta multivariée}
\label{sec:deltamethod}

Dans le cas univarié, on utilise la méthode delta pour évaluer la
distribution d'une fonction d'un estimateur, en supposant que la
distribution de cet estimateur est asymptotiquement normale de
variance connue.

Dans le cas multivarié, on estime la distribution d'une fonction d'un
vecteur d'estimateurs, dont la distribution asymptotique est normale
multivariée, avec une matrice de variance-covariance $\Sigma$.

On a donc, pour un estimateur convergent $\hat\theta_T$, en appliquant
le théorème central limite, le résultat suivant:
\begin{align}
  \sqrt{T} (\hat\theta_T - \theta_0) \stackrel{L}{\longrightarrow}
  \mathcal{N}(0,\Sigma).
\end{align}
On cherche la distribution d'une fonction $h(\hat\theta_T)$. On
développe cette fonction sous la forme d'une série de Taylor et en
conservant seulement les deux premiers termes:
\begin{align}
  h(\hat\theta_T) \approx h(\theta_0) + \nabla
  \left[h(\theta_0)\right]'\cdot(\hat\theta_T - \theta_0).
\end{align}
On a donc, après quelques manipulations, que la variance de la
fonction $h(\hat\theta_T)$ est approximativement
\begin{align}
  Var\left[h(\hat\theta_T) \right] \approx \nabla
  \left[h(\theta_0)\right]' \left(\frac{\Sigma}{T}\right) \nabla
  \left[h(\theta_0)\right].
\end{align}

La distribution de la fonction $h(\hat\theta_T)$ est alors
asymptotiquement
\begin{align}
  \label{eq:deltamethodmult}
  h(\hat\theta_T) \stackrel{L}{\longrightarrow}
  N\left(h(\theta_0),\nabla \left[h(\theta_0)\right]'
    \left(\frac{\Sigma}{T}\right) \nabla
    \left[h(\theta_0)\right]\right).
\end{align}



%%% Local Variables: 
%%% mode: latex
%%% TeX-master: "gabarit-maitrise"
%%% End: 
