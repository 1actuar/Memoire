\chapter{Tests statistiques}

\section{Test de normalité}
\label{sec:testnormalite}

On utilise un test de normalité afin de vérifier si l'on peut rejeter
l'hypothèse de normalité de la distribution des rendements. Comme
présenté au premier chapitre, c'est cette observation qui est à la
base de la recherche de modèles alternatifs, étant donné que cette
hypothèse est rejetée pour la plupart des séries financières. Il
existe plusieurs tests disponibles pour ce faire, dont ceux de
Shapiro-Wilk et d’Epps-Pulley.

\subsection{Test de Shapiro-Wilk}
\label{sec:test-de-shapiro}

Le test de Shapiro-Wilk \citep{shapiro1965analysis} est basé sur les
statistiques d'ordre de la distribution normale. Ce test est
particulièrement efficace même pour de petits échantillons $(T<20)$,
selon les auteurs.

On considère l'échantillon ordonné
$y_{(1)},y_{(2)},\ldots,y_{(T)},$. On évalue la variance $S^2$ de
l'échantillon puis la statistique $b$, où les valeurs $a_{T-i+1}$
proviennent de la \emph{Table 5} de \cite{shapiro1965analysis}:
\begin{enumerate}
\item Si la taille de l'échantillon $T$ est paire, alors
  $k=\frac{T}{2}$ et
  \begin{align}
    \label{eq:shapirowilkb1}
    b &= \sum_{i=1}^{k}a_{T-i+1}(y_{(T-i+1)}-y_{(i)}).
  \end{align}
\item Si la taille de l'échantillon $T$ est impaire, alors
  $k=\frac{T-1}{2}$ et
  \begin{align}
    \label{eq:shapirowilkb2}
    b &= a_{T}(y_{(T)}-y_{(1)}) + \ldots + a_{k+2}(y_{(k+2)}-y_{(k)}).
  \end{align}
\end{enumerate}

La statistique de Shapiro-Wilk $W = \frac{b}{S^2}$ suit une
distribution particulière dont on retrouve différents quantiles à la
\emph{Table 6} de \cite{shapiro1965analysis}. On rejette l'hypothèse
de la normalité lorsque la statistique $W$ est inférieure au seuil
critique déterminé à partir de cette dernière table.

\subsection{Test d’Epps-Pulley}
\label{sec:test-de-epps}

Le test d’Epps-Pulley \citep{epps1983test} est basé sur le carré de la
différence entre les fonctions caractéristiques empirique et théorique
de la distribution normale. Ce test est considéré comme un des plus
puissants pour un seuil de tolérance donné, pour de larges
échantillons. La statistique de test prend la forme suivante, où
$\overline{X}$ et $S$ sont respectivement la moyenne et l'écart-type
échantillonnal:
\begin{align}
  \label{eq:statistiqueEppsPulley}
  EP_T &= T\int_{-\infty}^{\infty} \left\vert\frac{1}{T}\sum_{j=1}^T
    \exp\left[it\frac{\left(X_j-\bar{X} \right)}{S} \right] - \exp
    \left[-\frac{t^2}{2} \right]\right\vert^2w(t)dt.
\end{align}

\cite{henze1990approximation} propose une procédure simple à
implémenter pour effectuer un test basé sur la statistique
d’Epps-Pulley $EP_T$, où la pondération de la différence quadratique
$w(\cdot)$ est remplacée par la densité de la loi normale centrée
réduite, ce qui permet d'accorder davantage d'importance aux
observations près de l'origine et aussi d'obtenir une forme
intégrable:
\begin{align}
  \label{eq:EppsPulley}
  EP_T &= \frac{2}{T} \sum_{1\leq\,j<k\leq\,T}
  \exp\left[-\frac{\left(X_j-X_k \right)^2}{2S^2}\right] - \sqrt{2}
  \sum_{j=1}^T \exp\left[-\frac{\left(X_j-\bar{X} \right)^2}{4S^2}
  \right]+\frac{T}{\sqrt{3}}+1.
\end{align}

Lorsque $T>10$, on calcule une version modifiée de la statistique:
\begin{equation}
  \label{eq:EppsPulleyMod}
  EP_T^{*} = \left(EP_T-\frac{0.365}{T}+\frac{1.34}{T^2}\right)\left(1+\frac{1.3}{T}\right).
\end{equation}

On obtient ensuite une approximation d'un quantile de la loi normale
centrée réduite:
\begin{align}
  \label{eq:ZEppsPulley}
  Z_T &= \gamma + \delta\ln\left(\left(EP_T^{*}-\xi
    \right)\left(\xi+\lambda-EP_T^{*} \right) \right).
\end{align}

Les constantes sont dérivées à l'équation \emph{4.1} de
\cite{henze1990approximation}:
\begin{align*}
  \gamma &\approx 3.55295 \\
  \delta &\approx 1.23062 \\
  \lambda &\approx 2.26664 \\
  \xi &\approx -0.020682.
\end{align*}

Puis, on compare cette statistique au quantile $Z_{1-\alpha}$ correspondant
au seuil de tolérance $\alpha$. Si $Z_n>Z_{1-\alpha}$
alors on rejette la normalité des données.

\section{Tests d'adéquation}
\label{sec:testnonparam}

Les tests d'adéquation vérifient l'ajustement à l'échantillon de la
fonction de répartition estimée soit:
\begin{itemize}
\item à partir de l'échantillon ayant servi à l'estimation du vecteur
  de paramètres
\item à partir d'un échantillon séparé, afin de vérifier, par exemple,
  si un modèle estimé à partir d'anciennes données s'applique toujours
  à de nouvelles informations.
\end{itemize}

On cherche à rejeter ou non une hypothèse concernant l'échantillon $Y$
et le vecteur de paramètres estimés:
\begin{itemize}
\item Si on considère le vecteur de paramètres comme étant estimé à
  partir du même échantillon sur lequel on effectue le test, alors on
  pose l'hypothèse composée suivante:
  \begin{equation}
    \label{eq:hypotheseadeqcomp}
    H_0: Y \sim F_Y(\hat\theta).
  \end{equation}
\item Si on considère le vecteur de paramètres comme étant connu ou
  estimé à partir d'un autre échantillon, alors on pose l'hypothèse
  simple suivante:
  \begin{equation}
    \label{eq:hypotheseadeq}
    H_0: Y \sim F_Y(\theta_0).
  \end{equation}
\end{itemize}

La différence entre ces deux hypothèses est le nombre de degrés de
liberté de la statistique de test. Les deux premiers tests présentés
sont des classiques, largement utilisés, mais dont l'application est
basée sur la fonction de répartition. Étant donné qu'on ne dispose que
d'une approximation de celle-ci, on utilisera alors un test basé sur
la fonction génératrice des moments, qui a une forme analytique.

\subsection{Test $\chi^2$ de Pearson}
\label{sec:testchi2}

Le test $\chi^2$ de Pearson \citep[ch. 8]{hogg1978introduction} est
basé sur une approximation multinomiale de la fonction de répartition
$F_X(x;\theta)$ pour laquelle on veut vérifier l'ajustement des
données échantillonnales $X_1,\ldots,X_n$. Pour ce faire, on divise le
domaine de la variable aléatoire $X$ en $k$ intervalles, appelés
classes. On associe la probabilité $p_i(\theta)$ que $X$ prend une
valeur dans l'intervalle $\left[c_{i-1},c_{i}\right]$. Cette
probabilité est évaluée à l'aide de la fonction de répartition
$F_X(x;\theta)$ de la variable aléatoire $X$:
\begin{equation}
  \label{eq:pmultinomiale}
  p_i(\theta) = F_X(c_{i};\theta) - F_X(c_{i-1};\theta),\qquad i = 1,\ldots,k.
\end{equation}

Soit $N_i$, le nombre de données de l'échantillon observées dans
l'intervalle défini précédemment. On définit alors la statistique $Q$
comme étant la somme, pour chaque intervalle, du carré de la
différence entre le nombre d'observations obtenues et espérées,
pondérée par cette dernière valeur. Cette quantité a approximativement
une distribution asymptotique $\chi^2(k-1)$ avec $k-1$ degrés de
liberté:
\begin{equation}
  \label{eq:statchi2}
  Q_{k-1} = \sum_{i=1}^{k} \frac{\left(N_i-np_i(\theta_0)\right)^2}{np_i(\theta_0)} \sim \chi^2(k-1).
\end{equation}

La statistique $Q_{k-1}$ peut être utilisée pour vérifier une
hypothèse simple, c'est-à-dire lorsque les paramètres sont connus:
\begin{align}
  \label{eq:hyp0-1}
  \mathcal{H}_0: \theta = \theta_0.
\end{align}

Cependant, comme les $q$ paramètres de la distribution $f_X$ doivent
être estimés pour évaluer les probabilités $p_i(\theta)$, on retrouve
alors une hypothèse composée:
\begin{align}
  \label{eq:hyp0-2}
  \mathcal{H}_0: F \in \lbrace F_{\theta} \rbrace, \theta \in \Omega.
\end{align}

On doit retrancher le même nombre de degrés de liberté à la
distribution asymptotique approximative de la statistique $Q$ qui
devient:
\begin{equation}
  \label{eq:statchi2param}
  Q_{k-q-1} = \sum_{i=1}^{k} \frac{\left(N_i-n{p}_i(\hat\theta)\right)^2}{n{p}_i(\hat\theta)} \sim \chi^2(k-q-1).
\end{equation}

Il est important de noter que ce test est approximatif, car pour
obtenir asymptotiquement la distribution $\chi^2$, les paramètres
doivent être estimés en minimisant la statistique $Q_{k-q-1}$
\eqref{eq:statchi2param}, qui devient alors une fonction objectif.

On rejette l'hypothèse composée $\mathcal{H}_0$ lorsque la condition
$Q_{k-q-1} > \chi^2_{1-\alpha}(k-q-1)$ est respectée. On effectue le
même test pour l'hypothèse simple, en posant $q=0$. Le point critique
de la distribution $\chi^2(k-q-1)$ est sélectionné selon le critère
\begin{equation}
  \label{eq:quantilechisq}
  \operatorname{Pr}\left(Q_{k-q-1}\leq \chi^2_{1-\alpha}(k-q-1)\right)=1-\alpha.
\end{equation}

\subsection{Test de Kolmogorov-Smirnov}
\label{sec:kolmogorovsmirnov}

Le test de Kolmogorov-Smirnov vérifie l'hypothèse simple
\eqref{eq:hypotheseadeq}. On définit la statistique $D_n$ comme étant
la valeur maximale de la distance entre les fonctions de répartition
empiriques $F_n(x)$ et celle $F_X(x)$ de la distribution exacte
spécifiée par l'hypothèse nulle $H_0$:
\begin{equation}
  \label{eq:statks}
  D_n=\sup_x |F_n(x)-F_X(x)|
\end{equation}
où
\begin{equation}
  \label{eq:repartemp}
  F_n(x)={1 \over n}\sum_{i=1}^n I_{X_i\leq x}.
\end{equation}

La statistique $\sqrt{n}D_n$ suit asymptotiquement une distribution de
Kolmogorov, selon \cite{wang2003evaluating}, dont la fonction de
répartition est définie comme suit:
\begin{align}
  \label{eq:repartkolmogorov}
  \operatorname{Pr}(\sqrt{n}D_n\leq x)&=1-2\sum_{k=1}^\infty (-1)^{k-1} e^{-2k^2 x^2}\\
  &=\frac{\sqrt{2\pi}}{x}\sum_{k=1}^\infty e^{-(2k-1)^2\pi^2/(8x^2)}.
\end{align}

On rejette l'hypothèse nulle $H_0$ lorsque la condition
$\sqrt{n}D_n>K_{1-\alpha}$ est respectée. On sélectionne le quantile
de cette distribution selon le critère suivant:
\begin{equation}
  \label{eq:quantilekolmogorov}
  \operatorname{Pr}(\sqrt{n}D_n \leq K_{1-\alpha})=1-\alpha.
\end{equation}

\subsection{Test de distance minimale basé sur la fonction génératrice
  des moments}
\label{sec:test-de-distance}

\cite{luong1987minimum} développent un ensemble de tests d'ajustement
pour l'hypothèse simple \eqref{eq:hypotheseadeq} basée sur une
transformation de la fonction de densité ou de répartition. Parmi
cette classe de tests, on retrouve celui d’Epps-Pulley, développé pour
vérifier l'hypothèse de normalité des données, tel que présenté à la
section \ref{sec:test-de-epps}. On retrouve aussi le test K-L de
\cite{feuerverger1981efficiency} basé sur les parties réelles et
imaginaires de la fonction caractéristique. Cependant, pour la
distribution de Laplace asymétrique généralisée, on ne peut effectuer
cette séparation. On préfèrera donc utiliser la fonction génératrice
des moments $M_Y(\xi)$ afin de construire la statistique de test
$D(F_T;f_{\theta_0})$. Comme on a déjà estimé les paramètres, on veut donc
vérifier une hypothèse simple. On se réfère à
\cite{KOUTROUVELIS01011980}, qui a développé le test K-L pour cette
situation. Cependant, on remplacera la fonction caractéristique par la
fonction génératrice des moments.

On considère un ensemble de fonctions $h_j(y), j=1,\ldots,K$. La
transformée de la fonction de répartition $F_Y(y)$ est donnée par le
vecteur $\mathbf{z}(F)=[z_1(F),\ldots,z_K(F)]$ :
\begin{align}
  \label{eq:4}
  z_j(F) &= \int_{-\infty}^{\infty} h_j(y) dF_Y(y)\nonumber\\
  &= E[h_j(y)]\nonumber\\
  &= M_Y(t_j),\quad j=1,\ldots,K.
\end{align}

En posant $h_j(y)=e^{t_j y}$, on définit $z_j(F)$ comme étant la
fonction génératrice des moments. On définit aussi la transformée de
la fonction de répartition empirique $F_T(y)$ par le vecteur
$\mathbf{z}(F_T)=[z_1(F_T),\ldots,z_K(F_T)]$, où
\begin{align}
  \label{eq:5}
  z_j(F_T) = \frac{1}{T} \sum_{t=1}^T e^{t_j y_t}.
\end{align}

La statistique de distance quadratique prend alors la forme suivante:
\begin{align}
  \label{eq:9}
  Td(F_n,F_\theta) &= T\left\{\mathbf{z}(F_n)-\mathbf{z}(F_\theta)
  \right\}^{\prime} \mathbf{Q}(F_\theta)
  \left\{\mathbf{z}(F_n)-\mathbf{z}(F_\theta) \right\} \nonumber\\
  &= \mathbf{v}_n^{\prime }\mathbf{Q}(F_\theta)\mathbf{v}_n.
\end{align}

On doit maintenant sélectionner une matrice $\mathbf{Q}(F_\theta)$ de
sorte que la distribution de la statistique $Td(F_n,F_\theta)$ soit
$\chi^2$. Comme $\mathbf{v}_n =
\sqrt{T}\left\{\mathbf{z}(F_n)-\mathbf{z}(F_\theta) \right\}$ suit
asymptotiquement une distribution normale multivariée de moyenne 0 et
de variance-covariance $\mathbf{\Sigma}$, on a, si la matrice
$\mathbf{\Sigma}$ est inversible et que l'on pose
$\mathbf{Q}=\mathbf{\Sigma}^{-1}$, que $\mathbf{v}_n^{\prime
}\mathbf{Q}(F_\theta)\mathbf{v}_n$ suit une distribution $\chi^2$ avec
$K$ degrés de liberté, dans le cadre d'une hypothèse simple.  Afin
d'évaluer la variance-covariance $\mathbf{\Sigma}$, on définit
l'espérance du produit des deux fonctions $h_j(y)$ et $h_k(y)$
:$z_{j,k}(F) = E[h_j(y)h_k(y)]$.

On décrit la matrice de variance-covariance $\mathbf{\Sigma}$ comme
suit:
\begin{align}
  \label{eq:14}
  \mathbf{\Sigma} &= \begin{bmatrix}
    \sigma_{1,1} & \cdots & \sigma_{1,k} \\
    \vdots & \ddots & \vdots \\
    \sigma_{j,1} & \cdots & \sigma_{j,k}
  \end{bmatrix}
\end{align}
où
\begin{align}
  \label{eq:10}
  \sigma_{j,k} &= z_{j,k}(F) - z_j(F)z_k(F)\nonumber\\
  &= M_Y(t_j+t_k) - M_Y(t_j)M_Y(t_k),\quad j,k=1,\ldots,K.
\end{align}

On peut maintenant évaluer la statistique $Td(F_n,F_\theta)$ et
effectuer le test. On rejette l'hypothèse $H_0$ lorsque la valeur de
celle-ci est supérieure au quantile correspondant au seuil $\alpha$ de
la distribution $\chi^2$ avec $K$ degrés de liberté:
$Td(F_n,F_\theta)>\chi^2_{1-\alpha}(K)$.


%%% Local Variables: 
%%% mode: latex
%%% TeX-master: "gabarit-maitrise"
%%% End: 
