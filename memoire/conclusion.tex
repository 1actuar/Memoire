\chapter*{Conclusion} % ne pas numéroter
\phantomsection\addcontentsline{toc}{chapter}{Conclusion} % dans TdM

%% Revenir sur le papier de Derman, résumé des avantages et
%% inconvénients du modèle, des méthodes d'estimation et d'évaluation
%% d'options

En guise de conclusion, j'aimerais tout d'abord effectuer un retour
sur les différents éléments introduits au début du premier chapitre,
concernant le risque de modélisation. Le modèle présenté n'en est pas
exempt, bien au contraire. Cependant, il
nécessite moins d'hypothèses restrictives que les autres modèles
présentés pour être valide, bien qu'il exige toujours l'indépendance
des observations. Il tient compte de la possibilité de sauts tout en
conservant une composante de mouvement aléatoire, ce qui décrit
adéquatement les observations empiriques à ce jour. Comme tout modèle
paramétrique, il reste dépendant du nombre et de la qualité des
données disponibles. Étant donné que l'utilisation d'algorithmes
d'optimisation numérique est inévitable, il subsiste un risque
important autour de l'estimation des paramètres et de l'approximation de la distribution. De plus, étant donné qu'il n'existe pas
de mesure neutre au risque unique, l'arbitrage de modèle reste
possible et doit être considéré. L'utilisation d'un échantillon de
données instables à travers le temps peut produire des résultats
inattendus, surtout au niveau de la distribution de la volatilité
historique, un aspect qui pourra être approfondi ultérieurement.
Enfin, subsiste toujours le risque d'erreurs de nature informatique
qui pourraient produire de faux résultats.

Ce retour permet de constater qu'il y a toujours place à
l'amélioration des outils développés. Entre autres, il pourrait être
pertinent d'étudier les différences entre le comportement à court et à long terme du modèle. En se basant sur la théorie de
l'utilité, on pourrait développer une meilleure approche pour
déterminer les paramètres de la distribution neutre au risque. Il
pourrait aussi être intéressant de développer des mesures de risque
cohérentes pour les processus de Lévy, notamment
avec les avancées de celles basées sur l'entropie. L'extension
multivariée de ce modèle n'a toujours pas été développée dans la
littérature, alors il pourrait être pertinent de s'y attarder, entre
autres pour étudier les titres indiciels et optimiser la composition
de portefeuilles. Enfin, il pourrait être intéressant d'aborder le
problème inverse de l'estimation des paramètres à partir des prix des
produits dérivés observés sur les marchés financiers.

%%% Local Variables: 
%%% mode: latex
%%% TeX-master: "gabarit-maitrise"
%%% End: 
