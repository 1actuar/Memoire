\chapter*{Introduction}
\phantomsection\addcontentsline{toc}{chapter}{Introduction} % inclure dans TdM

L'utilisation de la distribution de Laplace asymétrique généralisée
dans le cadre de la modélisation des rendements financiers est de plus
en plus courante dans les milieux académiques et pratiques. Entre
autres, elle présente une alternative intéressante au modèle de
Black-Scholes pour l'évaluation des options de type européennes. De
plus, elle peut être utilisée pour modéliser les rendements
obligataires à long terme et les variations du taux de
change. Cependant, la littérature actuelle présente peu d'outils qui
facilitent l'utilisation en pratique de cette distribution. La
principale motivation derrière le travail de recherche et de synthèse
présenté dans ce texte a été de développer certains outils qui
permettent l'estimation des paramètres de la distribution et
l'approximation de celle-ci. De plus, plusieurs tests statistiques
sont présentés afin d'évaluer la validité du modèle estimé et de
certaines contraintes linéaires qu'on pourrait lui appliquer.

Au chapitre 1, on présente les différents types de modèles financiers
ainsi que le risque associé à la modélisation. On introduit aussi
différents concepts théoriques entourant les rendements
financiers. Puis, on dresse un historique des modèles financiers ayant
mené à l'utilisation de la distribution de Laplace asymétrique
généralisée.

Au chapitre 2, on introduit les processus gamma et de Wiener qui sont
les deux composantes essentielles du processus de Laplace. Puis, on
présente les principales caractéristiques de la distribution de
Laplace asymétrique généralisée. Enfin, on présente quelques cas
particuliers de celle-ci et on fait le lien avec le modèle
variance-gamma de Madan et Seneta.

Au chapitre 3, on présente la méthode du point de selle qui permet
d'effectuer l'approximation de la densité et de la fonction de
répartition lorsqu'elles n'ont pas de forme analytique. On applique
ensuite cette méthode à la distribution de Laplace asymétrique
généralisée.

Au chapitre 4, on présente la méthode des moments généralisée ainsi
que son utilisation dans le cadre de l'estimation sous contraintes. On
présente ensuite certains tests d'hypothèse paramétriques.

Au chapitre 5, on présente la méthode des équations d'estimation
optimales ainsi qu'une version légèrement modifiée qui diminue la
quantité de calcul nécessaire.

Au chapitre 6, on applique les méthodes d'estimation des deux
chapitres précédents à la distribution de Laplace asymétrique
généralisée. On présente en premier lieu une méthode qui permet
d'obtenir un point de départ pour l'algorithme d'optimisation. Puis,
on détaille les résultats obtenus.

Au chapitre 7, on présente les tests de normalité de Shapiro-Wilk et
d'Epps-Pulley, puis ceux d'adéquation du $\chi^2$ de Pearson, celui de
Kolmogorov-Smirnov ainsi qu'un autre basé sur la fonction
génératrice des moments.

Au chapitre 8, on présente différentes méthodes pour l'évaluation
d'options. On présente d'abord quelques notions liées aux produits
dérivés, puis on détaille trois méthodes pouvant être utilisées avec
la distribution de Laplace asymétrique généralisée. Enfin, on présente
quelques particularités liées à certains titres financiers.

Au chapitre 9, on présente un exemple d'application des outils
développés aux chapitres 3 à 8 avec un ensemble de données.

Enfin, on présente en annexe certaines notions de théorie des
probabilités et de statistique.


%%% Local Variables: 
%%% mode: latex
%%% TeX-master: "gabarit-maitrise"
%%% End: 
