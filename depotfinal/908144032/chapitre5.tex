\chapter{Méthode de l'équation d'estimation
  optimale} % numéroté

Une équation d'estimation est une fonction des données de
l'échantillon et des paramètres d'un modèle qui spécifie de quelle
manière on doit procéder pour estimer ces derniers, lorsque la
distribution de la population est inconnue. Cette approche a un
avantage sur les méthodes de vraisemblance, car elle ne requiert pas
l'utilisation de la fonction de densité ou de répartition, mais
seulement les moments de la distribution. \citep{everitt2006cambridge}

\cite{crowder1986consistency} définit l'équation d'estimation sous une
forme générale, composée d'une matrice d'échelle $S(\theta;Y)$, un
vecteur de fonctions aléatoires $u(\theta;Y)$ et un vecteur de
pondération $\mathbf{w}(\theta;Y)$:
\begin{align}
  \label{eq:generalEE86}
  g(\theta;Y) = S^{-1}(\theta;Y) \sum_{t=1}^T u(\theta;Y)
  \mathbf{w}(\theta;Y).
\end{align}

En faisant abstraction du vecteur de pondération
$\mathbf{w}(\theta;Y)$, on retrouve une forme qui rappelle la méthode
des moments généralisée. La fonction d'estimation $u(\theta;Y)$
définit, dans ce cas, des conditions de moment.  Toutefois, en faisant
abstraction de la matrice $S(\theta;Y)$, on obtient une classe
d'équations d'estimation $g(\theta;Y)$ qui généralise plusieurs
méthodes connues, comme développé par \cite{crowder1987linear}:
\begin{align}
  \label{eq:generalEE87}
  g(\theta;Y) = \sum_{t=1}^T u(\theta;Y) \mathbf{w}(\theta;Y).
\end{align}

La propriété fondamentale de toute équation d'estimation est que son
espérance est nulle. L'équation d'estimation \eqref{eq:generalEE86},
et donc \eqref{eq:generalEE87}, produit des estimateurs sans biais:
\begin{align}
  \label{eq:EEsansbiais}
  E\left[ g(\theta;Y) \right] = 0.
\end{align}

Par analogie avec la méthode du maximum de vraisemblance, on pourra
alors la considérer comme une équation de quasi-score. On peut en
effet représenter l'équation précédente par celle du score
\eqref{eq:scoreEMV} donnée par l'espérance de la dérivée de la
fonction de log-vraisemblance par rapport au vecteur de paramètres
$\theta$:
\begin{align}
  \label{eq:scoreEMV}
  E\left[g\left(\theta;Y \right)^{EMV}\right] &= E\left[\frac{d \ln
      L\left(\theta;Y\right)}{d\theta}\right] = 0.
\end{align}

On désigne la moyenne et la variance de la distribution par les
notations $\mu\left(\theta\right)$ et $\sigma^2(\theta)$
respectivement.  On définit aussi les dérivées premières de la moyenne
et de l'écart-type par rapport au vecteur de paramètres:
\begin{subequations}\label{eq:derivmom}
  \begin{align}
    \mu^{\prime}\left( \theta \right) &= \frac{d \mu\left(\theta\right)}{d\theta} \label{eq:derivmom1}\\
    \sigma^{\prime}\left( \theta \right) &= \frac{d
      \sqrt{\sigma^2(\theta)}}{d\theta}.\label{eq:derivmom2}
  \end{align}
\end{subequations}

Dans ce chapitre, on considère une équation d'estimation quadratique
de la forme suivante:
\begin{equation}
  \label{eq:generalquad}
  g\left(\theta;Y \right) = \sum_{t=1}^n \left[ \mathbf{a}(\theta;y_t)(y_t-\mu\left(\theta\right)) + \mathbf{b}(\theta;y_t)\left((y_t-\mu\left(\theta\right))^2-\sigma^2\left(\theta\right) \right)\right].
\end{equation}

Le vecteur de pondération $\mathbf{w}(\theta;Y)$ de l'équation
\eqref{eq:generalEE87} est composé de deux fonctions déterministes:
$\mathbf{a}(\theta;Y)$ et $\mathbf{b}(\theta;Y)$:
\begin{align}
  \label{eq:vecteuroptimal.ab}
  \mathbf{w}(\theta;Y) &= \begin{bmatrix} \mathbf{a}(\theta;Y) \\
    \mathbf{b}(\theta;Y) \end{bmatrix}.
\end{align}

On démontre facilement que c'est une équation d'estimation en
calculant son espérance:
\begin{align}
  \label{eq:demogeneralquad-convergent}
  E\left[g\left(\theta;Y \right)\right] &= E\left[ \sum_{t=1}^n \left[ \mathbf{a}(\theta;y_t)(y_t-\mu\left(\theta\right)) + \mathbf{b}(\theta;y_t)\left((y_t-\mu\left(\theta\right))^2-\sigma^2\left(\theta\right) \right)\right]\right] \nonumber\\
  &=  \sum_{t=1}^n \left( \mathbf{a}(\theta;y_t)\cdot  E\left[y_t-\mu\left(\theta\right)\right] + \mathbf{b}(\theta;y_t)\cdot E\left[(y_t-\mu\left(\theta\right))^2-\sigma^2\left(\theta\right) \right]\right) \nonumber\\
  &= 0.
\end{align}

La forme quadratique regroupe certaines méthodes d'estimation bien
établies, recensées à la table \ref{tab:methodesquad}.
\begin{table}[!ht]
  
  \centering
  \begin{tabular}{ccc}
    \hline
    \textbf{Méthode} & $\mathbf{a}(\theta;y_t)$ & $\mathbf{b}(\theta;y_t)$ \\
    \hline
    Moindres carrés non pondérés & $\mu^{\prime}\left( \theta \right)$ & $0$ \\
    Maximum de quasi-vraisemblance & $\frac{\mu^{\prime}\left( \theta \right)}{ \sigma^2\left( \theta \right)}$ & $0$ \\
    Estimation gaussienne de Whittle
    \citep{fox1986large} & $\frac{\mu^{\prime}\left( \theta \right)}{ \sigma^2\left( \theta \right)}$ & $\frac{\sigma^{\prime}\left( \theta \right)}{ \sigma^2\left( \theta \right)^{3/2}}$ \\
    \hline
  \end{tabular}
  \caption{Méthodes d'estimation représentables par la forme quadratique \eqref{eq:generalquad}
  }\label{tab:methodesquad}
\end{table}

La méthode d'estimation gaussienne de Whittle présente un avantage par
rapport aux modèles linéaires conventionnels (moindres carrés pondérés
et maximum de quasi-vraisemblance) du fait qu'elle prend en
considération la variance comme critère d'optimisation et non
seulement comme pondération.

\cite{crowder1987linear} développe, à partir de cette dernière, 
une équation d'estimation optimale qui vise à
remplacer la méthode de quasi-vraisemblance basée sur la moyenne et la
variance jusqu'alors utilisée.

\section{Équation d'estimation optimale}
\label{sec:equationsquadopt}

Soit une équation d'estimation de la forme générale
\eqref{eq:generalEE87}. On a, par définition, que l'espérance de la
composante stochastique $u(\theta;Y_t)$ de celles-ci sous les vrais
paramètres $\theta_0$ est nulle. De plus, si l'on considère que les
données ne sont pas corrélées, l'espérance du produit de deux de ces
fonctions est nulle, sauf lorsqu'elle est évaluée au même point $Y_t$:
\begin{align}
  \label{eq:Crowder86-th4.1-def}
  E \left[u(\theta_0;Y_t) \right] &= 0 \\
  E \left[u(\theta_0;Y_s)u(\theta_0;Y_t) \right] &=
  \delta_{s,t}Var\left[u(\theta_0;Y_t)\right].
\end{align}

Les matrices non singulières $M(\theta_0) \mbox{ et } V(\theta_0)$
représentent respectivement, pour l'équation d'estimation, l'espérance
de sa dérivée par rapport à $\theta$ et l'inverse de sa
variance-covariance, évaluée à la vraie valeur des paramètres
$\theta_0$:
\begin{subequations}\label{eq:Crowder86-th3.3-def}
  \begin{align}
    M(\theta_0;Y) &= \sum_{t=1}^{T} \mathbf{w}(\theta_0) E\left[u^{\prime}(\theta_0;Y_t) \right] \label{eq:Crowder86-th3.3-def-1}\\
    V(\theta_0;Y) &= \sum_{t=1}^{T} \mathbf{w}(\theta_0)
    Var\left[u(\theta_0;Y_t)\right]
    \mathbf{w}^{T}(\theta_0)\label{eq:Crowder86-th3.3-def-2}.
  \end{align}
\end{subequations}

La variance asymptotique de l'estimateur $\hat\theta$ prend alors la
forme suivante, en utilisant la méthode delta multivariée:
\begin{align}
  \label{eq:VarAsymptEstEE}
  Var\left[\hat\theta\right] &=
  M^{-1}(\theta_0)V(\theta_0)\left[M^{-1}(\theta_0)\right]^{T}.
\end{align}

On désire obtenir des équations d'estimation optimales de telle sorte
que la variance soit minimale. Le \emph{théorème 4.1} de
\cite{crowder1986consistency} permet d'établir quel vecteur de
pondération $\mathbf{w}^{*}(\theta)$ utiliser.

Selon ce théorème, lorsqu'on utilise le vecteur de pondération optimal
\eqref{eq:Crowdervecteuroptimal}, on obtient que la dérivée de la
fonction d'estimation $M^{*}(\theta)$ soit égale à l'inverse de la
matrice de variance-covariance $V^{*}(\theta)$:
\begin{align}
  \label{eq:Crowdervecteuroptimal}
  \mathbf{w}^{*}(\theta) &= \left\{E \left[u^{\prime}(\theta_0;Y_t)
    \right]\right\}^{T}\left\{Var\left[u(\theta;Y_t)\right]
  \right\}^{-1}.
\end{align}

La variance asymptotique \eqref{eq:VarAsymptEstEE} du vecteur de
paramètres optimal $\hat\theta^{*}$ qui résout l'équation d'estimation
$g^{*}(\theta)$ se simplifie alors sous la forme:
$\left[V^{*}(\theta_0;y_t)\right]^{-1}$
\begin{align}
  \label{eq:13}
  M^{*}(\theta_0) &= \sum_{t=1}^{T} \left\{E
    \left[u^{\prime}(\theta_0;Y_t)
    \right]\right\}^{T}\left\{Var\left[u(\theta_0;Y_t)\right]
  \right\}^{-1} E\left[u^{\prime}(\theta_0;Y_t) \right] \\
  V^{*}(\theta_0) &= \sum_{t=1}^{T} \left\{E
    \left[u^{\prime}(\theta_0;Y_t)
    \right]\right\}^{T}\left\{Var\left[u(\theta_0;Y_t)\right]
  \right\}^{-1} Var\left[u(\theta_0;Y_t)\right] \nonumber\\
  &\quad \times \left[\left\{E \left[u^{\prime}(\theta_0;Y_t)
      \right]\right\}^{T}\left\{Var\left[u(\theta_0;Y_t)\right]
    \right\}^{-1}\right]^{T} \nonumber\\
  &= \sum_{t=1}^{T} \left\{E \left[u^{\prime}(\theta_0;Y_t)
    \right]\right\}^{T}\left\{Var\left[u(\theta_0;Y_t)\right]
  \right\}^{-1} E\left[u^{\prime}(\theta_0;Y_t) \right] \nonumber\\
  \Leftrightarrow M^{*}(\theta_0) &= V^{*}(\theta_0) \nonumber\\
  \Leftrightarrow Var(\hat\theta^{*}) &= \left[M^{*}(\theta_0)\right]^{-1} V^{*}(\theta_0) \left[\left[M^{*}(\theta_0)\right]^{-1}\right]^{T} \nonumber\\
  &= \left[V^{*}(\theta_0)\right]^{-1}.
\end{align}

On considère la fonction $u(\theta;Y_t)$ de l'équation quadratique
\eqref{eq:generalquad} , dont on cherche le vecteur de pondération
optimal correspondant selon la proposition
\eqref{eq:Crowdervecteuroptimal}:
\begin{align}
  \label{eq:15}
  u(\theta;Y_t) &= \begin{bmatrix}
    Y_t-\mu(\theta)\\
    (Y_t-\mu(\theta))^2-\sigma^2(\theta)
  \end{bmatrix}.
\end{align}

On évalue l'espérance, sous le vecteur de vrais paramètres $\theta_0$,
de la dérivée par rapport à $\theta$ de la fonction $u(\theta;Y)$:
\begin{align}
  \label{eq:16}
  E\left[u^{\prime}(\theta;Y_t)\right] &= \begin{bmatrix}
    E \left[-\mu^{\prime}(\theta)\right] \\
    E
    \left[-2(Y-\mu(\theta))\mu^{\prime}(\theta)-2\sqrt{\sigma^2(\theta)}\sigma^{\prime}(\theta)\right]
  \end{bmatrix} \nonumber\\
  &= \begin{bmatrix}
    -\mu^{\prime}(\theta) \\
    -2 \sqrt{\sigma^2(\theta)}\sigma^{\prime}(\theta).
  \end{bmatrix}.
\end{align}

Puis, on évalue la variance-covariance du vecteur $u(\theta)$. Pour ce
faire, on devra évaluer séparément chaque élément composant cette
matrice:
\begin{align}
  \label{eq:matvcov-u-EE}
  Var\left[u(\theta;Y_t) \right] &=
  \begin{bmatrix}
    Var\left[Y-\mu(\theta)\right] & Cov\left[Y-\mu(\theta),(Y-\mu(\theta))^2-\sigma^2(\theta) \right] \\
    Cov\left[Y-\mu(\theta),(Y-\mu(\theta))^2-\sigma^2(\theta) \right]
    & Var\left[(Y-\mu(\theta))^2-\sigma^2(\theta)\right]
  \end{bmatrix}.
\end{align}

On définit les coefficients d'asymétrie et d'aplatissement normalisés,
ainsi qu'une constante $\gamma_3(\theta)$ qui en découle, afin de
simplifier les expressions qui seront obtenues conséquemment:
\begin{subequations}\label{eq:momentsupgamma}
  \begin{align}
    \gamma_1\left(\theta\right) &= \frac{m_3\left(\theta\right)}{m_2\left(\theta\right)^{3/2}} \nonumber\\
    &= \frac{E\left[\left(Y-\mu\left(\theta\right)\right)^3\right]}{\sigma^2\left(\theta\right)^{3/2}}  \label{eq:momentsupgamma1}\\
    \gamma_2\left(\theta\right) &= \frac{m_4\left(\theta\right)}{m_2\left(\theta\right)^{2}} - 3 \nonumber\\
    &=\frac{E\left[\left(Y-\mu\left(\theta\right)\right)^4\right]}{\sigma^2\left(\theta\right)^2}-3 \label{eq:momentsupgamma2}\\
    \gamma_3(\theta) &= \gamma_2\left(\theta\right) + 2 -
    \gamma_1\left(\theta\right)^2. \label{eq:momentsupgamma3}
  \end{align}
\end{subequations}

On développe les différentes composantes de la matrice:
\eqref{eq:matvcov-u-EE}
\begin{align}
  \label{eq:18}
  Var\left[Y-\mu(\theta)\right] &= E\left[(Y-\mu(\theta))^2\right] -
  E\left[Y-\mu(\theta)\right]^2 \nonumber\\
  &=\sigma^2(\theta)\\
  Cov\left[Y-\mu(\theta),(Y-\mu(\theta))^2-\sigma^2(\theta) \right] &=
  E\left[\left(Y-\mu(\theta)\right)\left((Y-\mu(\theta))^2-\sigma^2(\theta)\right)
  \right] \nonumber\\ &\quad - E\left[Y-\mu(\theta) \right]
  E\left[(Y-\mu(\theta))^2-\sigma^2(\theta) \right] \nonumber\\
  &= E\left[(Y-\mu(\theta))^3 - \sigma^2(\theta)(Y-\mu(\theta))\right] \nonumber\\
  &= \sigma^2(\theta)^{3/2}\gamma_1(\theta)\\
  Var\left[(Y-\mu(\theta))^2-\sigma^2(\theta)\right] &=
  E\left[\left((Y-\mu(\theta))^2-\sigma^2(\theta)
    \right)\left((Y-\mu(\theta))^2-\sigma^2(\theta) \right) \right]\nonumber\\
  &= E\left[(Y-\mu(\theta))^4 - 2\sigma^2(\theta)(Y-\mu(\theta))^2 +
    \sigma^2(\theta)^2 \right] \nonumber\\
  &= \sigma^2(\theta)^2\left(\gamma_2(\theta)+3\right) - 2\sigma^2(\theta)^2 + \sigma^2(\theta)^2 \nonumber\\
  &=\sigma^2(\theta)^2\left(\gamma_2(\theta)+2 \right).
\end{align}

On obtient alors la matrice de variance-covariance de la fonction
$u(\theta)$, dont on évalue par la suite l'inverse:
\begin{align}
  \label{eq:19}
  Var\left[u(\theta;Y_t) \right] &= \begin{bmatrix}
    \sigma^2(\theta) & \sigma^2(\theta)^{3/2}\gamma_1(\theta)\\
    \sigma^2(\theta)^{3/2}\gamma_1(\theta) &
    \sigma^2(\theta)^2\left(\gamma_2(\theta)+2 \right)
  \end{bmatrix} \\
  Var^{-1}\left[u(\theta;Y_t) \right] &=
  \frac{1}{\sigma^2(\theta)^3\gamma_3(\theta)}\begin{bmatrix}
    \sigma^2(\theta)^2\left(\gamma_2(\theta)+2 \right) & -\sigma^2(\theta)^{3/2}\gamma_1(\theta)\\
    -\sigma^2(\theta)^{3/2}\gamma_1(\theta) & \sigma^2(\theta)
  \end{bmatrix}.
\end{align}

On peut enfin évaluer l'expression \eqref{eq:Crowdervecteuroptimal} à
l'aide des résultats précédents:
\begin{align*}
  \mathbf{w}^{*}(\theta) &= \begin{bmatrix} -\mu^{\prime}(\theta) & -2
    \sqrt{\sigma^2(\theta)}\sigma^{\prime}(\theta).
  \end{bmatrix} \times
  \frac{1}{\sigma^2(\theta)^3\gamma_3(\theta)} \begin{bmatrix}
    \sigma^2(\theta)^2\left(\gamma_2(\theta)+2 \right) & -\sigma^2(\theta)^{3/2}\gamma_1(\theta)\\
    -\sigma^2(\theta)^{3/2}\gamma_1(\theta) & \sigma^2(\theta)
  \end{bmatrix}.
\end{align*}

On obtient alors le vecteur de pondération optimal
\eqref{eq:vecteuroptimal.ab} formé des fonctions
$\mathbf{a}^{*}(\theta)$ et $\mathbf{b}^{*}(\theta)$:
\begin{subequations}\label{eq:coefficientscrowder}
  \begin{align}
    \mathbf{a}^{*}(\theta) &= \frac{\left\{ -\left( \gamma_2\left(\theta\right)+2 \right) \mu^{\prime}\left( \theta \right) + 2\gamma_1\left(\theta\right) \sigma^{\prime}\left( \theta \right) \right\}}{\sigma^2\left(\theta\right)\gamma_3(\theta)} \label{eq:acrowder}\\
    \mathbf{b}^{*}(\theta) &=
    \frac{\gamma_1\left(\theta\right)\mu^{\prime}\left( \theta
      \right)-2\sigma^{\prime}\left( \theta \right)}{\sigma^2\left(
        \theta \right)^{3/2}\gamma_3(\theta)}. \label{eq:bcrowder}
  \end{align}
\end{subequations}

On obtient aisément les matrices $M^{\star}\left(\theta\right) \mbox{
  et } V^{\star}\left(\theta\right)$ à partir de leur définition
\eqref{eq:Crowder86-th3.3-def}:
\begin{align}
  M^{\star}\left(\theta\right) &= V^{\star}\left(\theta\right) \nonumber\\
  &= T \cdot \sigma(\theta)^{-2} \left\{ \left(\mu^{\prime}\left(
        \theta \right)\right)\left(\mu^{\prime}\left( \theta
      \right)\right)^T+\gamma_3(\theta)^{-1}\left(\gamma_1\left(\theta\right)\mu^{\prime}\left(
        \theta \right)-2\sigma^{\prime}\left( \theta
      \right)\right)\left(\gamma_1\left(\theta\right)\mu^{\prime}\left(
        \theta \right)-2\sigma^{\prime}\left( \theta \right)\right)^T
  \right\}. \label{eq:Moptimalestimetheta}
\end{align}

Afin d'obtenir la solution optimale $\theta^{\star}$, on définit la
fonction objectif $\Lambda\left(\theta;\mathbf{y}_T\right)$:
\begin{align}
  \label{eq:eqnobjectifEE}
  \Lambda\left(\theta;\mathbf{y}_T\right) &=
  g\left(\theta;\mathbf{y}_T \right) W\left( \theta;\mathbf{y}_T
  \right) g\left( \theta;\mathbf{y}_T \right)^T.
\end{align}

$W\left( \theta ;\mathbf{y}_T \right)$ est une matrice définie
positive. En premier lieu, on suggère d'utiliser la matrice
identité. Puis, on peut raffiner l'estimation en lui substituant la
matrice $V^{\star}(\hat\theta)$ de manière itérative, de la même
manière qu'avec la méthode des moments généralisée itérative, à la
section \ref{sec:GMMtwostep}.

\section{Équation d'estimation optimale modifiée}
\label{sec:eqoptmodif}

Il est possible que, pour certaines distributions, les expressions
formant le vecteur de pondération optimal $\mathbf{w}^{*}(\theta)$
soient particulièrement complexes. On pourra toujours les utiliser,
cependant, il peut être intéressant de développer une approximation de
celles-ci. Pour ce faire, on substitue les valeurs théoriques des
coefficients d'asymétrie et d'aplatissement par une valeur estimée à
partir de l'échantillon:
\begin{subequations}\label{eq:vecteurcrowdermod}
  \begin{align}
    \mathbf{a}_{mod}^{*}(\theta) &= \frac{\left\{ -\left( \hat\gamma_2\left(\mathbf{y}_T\right)+2 \right) \mu^{\prime}\left( \theta \right) + 2\hat\gamma_1\left(\mathbf{y}_T\right) \sigma^{\prime}\left( \theta \right) \right\}}{\sigma^2\left(\theta\right)\hat\gamma_3(\mathbf{y}_T)} \label{eq:acrowdermod}\\
    \mathbf{b}_{mod}^{*}(\theta) &=
    \frac{\hat\gamma_1\left(\mathbf{y}_T\right)\mu^{\prime}\left(
        \theta \right)-2\sigma^{\prime}\left( \theta
      \right)}{\sigma^2\left( \theta
      \right)^{3/2}\hat\gamma_3(\mathbf{y}_T)} \label{eq:bcrowdermod}
  \end{align}

  où
  \begin{align}
    \label{eq:coeffemp}
    \hat\gamma_1\left(\mathbf{y}_T\right) &= \frac{\sum_{t=1}^{T}(y_t-\overline{y})^3}{\left(\sum_{t=1}^{T}(y_t-\overline{y})^2\right)^{3/2}} \\
    \hat\gamma_2\left(\mathbf{y}_T\right) &=
    \frac{\sum_{t=1}^{T}(y_t-\overline{y})^4}{\left(\sum_{t=1}^{T}(y_t-\overline{y})^2\right)^{2}} \\
    \hat\gamma_3\left(\mathbf{y}_T\right) &=
    \hat\gamma_2\left(\mathbf{y}_T\right) + 2 -
    \hat\gamma_1\left(\mathbf{y}_T\right)^2.
  \end{align}
\end{subequations}

On pourra ainsi réduire considérablement la taille des expressions à
évaluer tout en conservant les propriétés asymptotiques des
estimateurs, puisque l'espérance de chacune des statistiques utilisées
est égale à la valeur que l'on a remplacée.

Au prochain chapitre, nous appliquerons la méthode des moments
généralisée et la méthode de l'équation d'estimation optimale à la
distribution de Laplace asymétrique généralisée.

%%% Local Variables: 
%%% mode: latex
%%% TeX-master: "gabarit-maitrise"
%%% End: 
