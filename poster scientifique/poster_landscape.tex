\documentclass[landscape,final,a0paper,fontscale=0.285]{baposter}

\usepackage{calc}
\usepackage{graphicx}
\usepackage{amsmath}
\usepackage{amssymb}
\usepackage{relsize}
\usepackage{multirow}
\usepackage{rotating}
\usepackage{bm}
\usepackage{url}
\usepackage[francais]{babel}
\usepackage{graphicx}
\usepackage{multicol}
\usepackage{amsmath}
\usepackage{amsfonts}
\usepackage[utf8]{inputenc}
\usepackage[T1]{fontenc}
%\usepackage{times}
%\usepackage{helvet}
%\usepackage{bookman}
\usepackage{palatino}

\newcommand{\captionfont}{\footnotesize}

\graphicspath{{images/}{../images/}}
\usetikzlibrary{calc}

%%%%%%%%%%%%%%%%%%%%%%%%%%%%%%%%%%%%%%%%%%%%%%%%%%%%%%%%%%%%%%%%%%%%%%%%%%%%%%%%
%%%% Some math symbols used in the text
%%%%%%%%%%%%%%%%%%%%%%%%%%%%%%%%%%%%%%%%%%%%%%%%%%%%%%%%%%%%%%%%%%%%%%%%%%%%%%%%

%%%%%%%%%%%%%%%%%%%%%%%%%%%%%%%%%%%%%%%%%%%%%%%%%%%%%%%%%%%%%%%%%%%%%%%%%%%%%%%%
% Multicol Settings
%%%%%%%%%%%%%%%%%%%%%%%%%%%%%%%%%%%%%%%%%%%%%%%%%%%%%%%%%%%%%%%%%%%%%%%%%%%%%%%%
\setlength{\columnsep}{1.5em}
\setlength{\columnseprule}{0mm}

%%%%%%%%%%%%%%%%%%%%%%%%%%%%%%%%%%%%%%%%%%%%%%%%%%%%%%%%%%%%%%%%%%%%%%%%%%%%%%%%
% Save space in lists. Use this after the opening of the list
%%%%%%%%%%%%%%%%%%%%%%%%%%%%%%%%%%%%%%%%%%%%%%%%%%%%%%%%%%%%%%%%%%%%%%%%%%%%%%%%
\newcommand{\compresslist}{%
\setlength{\itemsep}{1pt}%
\setlength{\parskip}{0pt}%
\setlength{\parsep}{0pt}%
}

%%%%%%%%%%%%%%%%%%%%%%%%%%%%%%%%%%%%%%%%%%%%%%%%%%%%%%%%%%%%%%%%%%%%%%%%%%%%%%
%%% Begin of Document
%%%%%%%%%%%%%%%%%%%%%%%%%%%%%%%%%%%%%%%%%%%%%%%%%%%%%%%%%%%%%%%%%%%%%%%%%%%%%%

\begin{document}

%%%%%%%%%%%%%%%%%%%%%%%%%%%%%%%%%%%%%%%%%%%%%%%%%%%%%%%%%%%%%%%%%%%%%%%%%%%%%%
%%% Here starts the poster
%%%---------------------------------------------------------------------------
%%% Format it to your taste with the options
%%%%%%%%%%%%%%%%%%%%%%%%%%%%%%%%%%%%%%%%%%%%%%%%%%%%%%%%%%%%%%%%%%%%%%%%%%%%%%
% Define some colors

%\definecolor{lightblue}{cmyk}{0.83,0.24,0,0.12}
\definecolor{lightblue}{rgb}{0.145,0.6666,1}
\hyphenation{resolution occlusions}
%%
\begin{poster}%
  % Poster Options
  {
  % Show grid to help with alignment
  grid=false,
  % Column spacing
  colspacing=1em,
  % Color style
  bgColorOne=white,
  bgColorTwo=white,
  borderColor=lightblue,
  headerColorOne=black,
  headerColorTwo=lightblue,
  headerFontColor=white,
  boxColorOne=white,
  boxColorTwo=lightblue,
  % Format of textbox
  textborder=roundedleft,
  % Format of text header
  eyecatcher=true,
  headerborder=closed,
  headerheight=0.1\textheight,
%  textfont=\sc, An example of changing the text font
  headershape=roundedright,
  headershade=shadelr,
  headerfont=\Large\bf\textsc, %Sans Serif
  textfont={\setlength{\parindent}{1.5em}},
  boxshade=plain,
%  background=shade-tb,
  background=plain,
  linewidth=2pt
  }
  % Eye Catcher
  {\includegraphics[height=5em]{images/ullong.pdf}}
  % Title
  {\bf\textsc{Distribution de Laplace asymétrique généralisée}\vspace{0.5em}}
  % Authors
  {\textsc{François Pelletier, Andrew Luong} \\ \small{École d'actuariat, Université Laval}}
  % University logo
  {% The makebox allows the title to flow into the logo, this is a hack because of the L shaped logo.
   
  }

%%%%%%%%%%%%%%%%%%%%%%%%%%%%%%%%%%%%%%%%%%%%%%%%%%%%%%%%%%%%%%%%%%%%%%%%%%%%%%
%%% Now define the boxes that make up the poster
%%%---------------------------------------------------------------------------
%%% Each box has a name and can be placed absolutely or relatively.
%%% The only inconvenience is that you can only specify a relative position 
%%% towards an already declared box. So if you have a box attached to the 
%%% bottom, one to the top and a third one which should be in between, you 
%%% have to specify the top and bottom boxes before you specify the middle 
%%% box.
%%%%%%%%%%%%%%%%%%%%%%%%%%%%%%%%%%%%%%%%%%%%%%%%%%%%%%%%%%%%%%%%%%%%%%%%%%%%%%
    %
    % A coloured circle useful as a bullet with an adjustably strong filling
    \newcommand{\colouredcircle}{%
      \tikz{\useasboundingbox (-0.2em,-0.32em) rectangle(0.2em,0.32em); \draw[draw=black,fill=lightblue,line width=0.03em] (0,0) circle(0.18em);}}

%%%%%%%%%%%%%%%%%%%%%%%%%%%%%%%%%%%%%%%%%%%%%%%%%%%%%%%%%%%%%%%%%%%%%%%%%%%%%%
  \headerbox{Introduction}{name=problem,column=0,row=0}{
%%%%%%%%%%%%%%%%%%%%%%%%%%%%%%%%%%%%%%%%%%%%%%%%%%%%%%%%%%%%%%%%%%%%%%%%%%%%%%
    \small{On définit le prix au temps $t$ d'un titre financier $S(t)$ et le rendement cumulé sur ce titre
\begin{equation*}L_t = R_1 + \ldots + R_{t} = \log S(t) - \log S(0). \end{equation*}
On veut modéliser $R_t$ par la distribution de Laplace asymétrique généralisée, aussi appelée variance-gamma. L'estimation paramétrique se fera à partir de la méthode GMM. 
On pourra ainsi obtenir la distribution de $L_t$ par convolution, afin d'évaluer le prix d'options européennes.}
 }

%%%%%%%%%%%%%%%%%%%%%%%%%%%%%%%%%%%%%%%%%%%%%%%%%%%%%%%%%%%%%%%%%%%%%%%%%%%%%%
  \headerbox{Simulation}{name=contribution,column=1,row=0,bottomaligned=problem}{
%%%%%%%%%%%%%%%%%%%%%%%%%%%%%%%%%%%%%%%%%%%%%%%%%%%%%%%%%%%%%%%%%%%%%%%%%%%%%%
    \small{Le processus de Laplace $Y(t)$ peut être représenté comme la différence de deux processus gamma $G(t)$ \cite{kotz2001laplace}.
   \begin{equation*} Y \stackrel{d}{=} \theta + \frac{\sigma}{\sqrt{2}} \left( \frac{1}{\kappa} G_1 - \kappa G_2 \right)
    \end{equation*} où $ G_1,G_2 \sim \Gamma\left(\alpha=\tau,\beta=1 \right)$.
Il suffit donc de simuler deux réalisations de cette variable
aléatoire pour obtenir une réalisation de la distribution Laplace
asymétrique généralisée.
}
}
%%%%%%%%%%%%%%%%%%%%%%%%%%%%%%%%%%%%%%%%%%%%%%%%%%%%%%%%%%%%%%%%%%%%%%%%%%%%%%
\headerbox{Estimation du modèle}{name=results,column=2,span=1,row=0}{
  %%%%%%%%%%%%%%%%%%%%%%%%%%%%%%%%%%%%%%%%%%%%%%%%%%%%%%%%%%%%%%%%%%%%%%%%%%%%%%
    \small{On utilise les conditions de moment basées sur l'espérance et la variance
    \begin{equation*}
  g(Y_t,\theta) =
  \left[\begin{array}[]{c}
    \bar{Y} - \left(\theta+\mu\tau\right) \\
    \frac{1}{T}\sum_{t=1}^T \left(Y_t - \bar{Y}  \right)^2 - \tau\left(\sigma^2+\mu^2\right)
  \end{array}\right]
\end{equation*}
La matrice optimale $W$ est estimée en utilisant la méthode du GMM itératif \cite{hall2005generalized}:
    \begin{align*}
      \hat{W}_{(1)} &=  \bigg(\frac{1}{T}\sum_{t=1}^T g(Y_t,\hat\theta_{(0)})g(Y_t,\hat\theta_{(0)})^{\prime}\bigg)^{-1} \\
      \hat{W}_{(i)} &=  \bigg(\frac{1}{T}\sum_{t=1}^T g(Y_t,\hat\theta_{(i-1)})g(Y_t,\hat\theta_{(i-1)})^{\prime}\bigg)^{-1}
    \end{align*}}
}
%%%%%%%%%%%%%%%%%%%%%%%%%%%%%%%%%%%%%%%%%%%%%%%%%%%%%%%%%%%%%%%%%%%%%%%%%%%%%%
  \headerbox{Graphique}{name=references,column=3,span=1,bottomaligned=results}{
%%%%%%%%%%%%%%%%%%%%%%%%%%%%%%%%%%%%%%%%%%%%%%%%%%%%%%%%%%%%%%%%%%%%%%%%%%%%%%
    \includegraphics[width=64mm,height=56mm]{GraphiqueAffiche1}
}

%%%%%%%%%%%%%%%%%%%%%%%%%%%%%%%%%%%%%%%%%%%%%%%%%%%%%%%%%%%%%%%%%%%%%%%%%%%%%%
  \headerbox{Processus de Laplace}{name=references,column=0,above=bottom}{
%%%%%%%%%%%%%%%%%%%%%%%%%%%%%%%%%%%%%%%%%%%%%%%%%%%%%%%%%%%%%%%%%%%%%%%%%%%%%%
    \small{Le processus de Laplace $Y(t)$ est un processus de Lévy subordonné, c'est-à-dire qu'il est formé de deux processus de Lévy, l'un qui détermine l'amplitude des évènements (sauts) et l'autre qui détermine le temps entre les évènements.
    Le premier est un processus de Wiener $W(t+\tau)-W(t) \sim N(0,\tau\sigma^2)$ et le second un processus gamma $G(t+\tau)-G(t) \sim \Gamma(\tau,\nu\tau)$.
    \begin{equation*}Y(t) = W(G(t))\end{equation*}}
  }
%%%%%%%%%%%%%%%%%%%%%%%%%%%%%%%%%%%%%%%%%%%%%%%%%%%%%%%%%%%%%%%%%%%%%%%%%%%%%%
  \headerbox{Méthode GMM}{name=questions,column=1,span=1,aligned=references,above=bottom}{
%%%%%%%%%%%%%%%%%%%%%%%%%%%%%%%%%%%%%%%%%%%%%%%%%%%%%%%%%%%%%%%%%%%%%%%%%%%%%%
  La méthode GMM a pour objectif d'estimer les paramètres $\theta$ d'une distribution en minimisant une norme quadratique
\begin{equation*}
  \|\hat{m}\left(\theta\right) \|_W^2 = \hat{m}\left(\theta\right)^{\prime} W \hat{m}\left(\theta\right)
\end{equation*}
pondérée par une matrice définie positive $W$ et où $\hat{m}\left(\theta\right) = \left(\frac{1}{T}\sum_{t=1}^T g(Y_t,\theta)\right)$ est la moyenne empirique des conditions de moment $g(Y_t,\theta)$.
  }
%%%%%%%%%%%%%%%%%%%%%%%%%%%%%%%%%%%%%%%%%%%%%%%%%%%%%%%%%%%%%%%%%%%%%%%%%%%%%%
  \headerbox{Références}{name=references,column=2,span=2,aligned=references,above=bottom}{
%%%%%%%%%%%%%%%%%%%%%%%%%%%%%%%%%%%%%%%%%%%%%%%%%%%%%%%%%%%%%%%%%%%%%%%%%%%%%%
  \bibliographystyle{plain}	% (uses file "plain.bst")
  \bibliography{biblio}		% expects file "myrefs.bib"
  }
%%%%%%%%%%%%%%%%%%%%%%%%%%%%%%%%%%%%%%%%%%%%%%%%%%%%%%%%%%%%%%%%%%%%%%%%%%%%%%
\headerbox{Prix d'options}{name=speed,column=2,span=2,row=0,below=results,above=references}{
  %%%%%%%%%%%%%%%%%%%%%%%%%%%%%%%%%%%%%%%%%%%%%%%%%%%%%%%%%%%%%%%%%%%%%%%%%%%%%%
  \small{L'évaluation du prix d'un option de vente européenne équivaut à calculer la valeur espérée de la réclamation contingente définie par le contrat
    \begin{equation}
      P(Y(t),T-t) = B(t,T) \int_{0}^{K} (K-Y(T)) d\hat{F}_t(Y(T))
    \end{equation}
    où $B(t,T)$ est la valeur d'une obligation zéro-coupon au taux sans
    risque $r$ d'échéance $T$ et $K$ le prix d'exercice. Cette évaluation doit de faire sous une mesure neutre au risque, selon laquelle les investisseurs n'exigent pas une prime de risque. Cette mesure doit répondre à la propriété martingale. Pour la distribution étudiée, ceci se fait par une modification du paramètre de dérive $\theta$ en $\theta^{\star}$
    \begin{align*}
      e^{rt} = E\left[exp(L_t) \right] = M_{Y_1+\ldots+Y_t}(1) = (\phi_{Y}(-i))^t = \left(\frac{e^{\theta^{\star}}}{(1-\mu-\sigma^2/2)^{\tau}} \right)^t \Rightarrow \theta^{\star} = r + \tau\log(1-\mu-\frac{\sigma^2}{2})
    \end{align*}
    On pourra ensuite évaluer la valeur du prix de l'option de vente avec la formule de Heston (1993) décrite dans \cite{epps2007pricing}
    \begin{align*}
      P(Y(t),T-t) &= B(t,T)K \int_{0}^{K} d{F}_t(Y(T)) - B(t,T)\int_{0}^{K} Y(T)\cdot d{F}_t(Y(T)) = B(t,T)K \hat{F}_t(K) - Y(t)\hat{G}_t(K)
    \end{align*}
où $\hat{G}_t(K)$ est la fonction de répartition de la transformée d'Esscher (h=1) de la mesure neutre au risque.
  }}
%%%%%%%%%%%%%%%%%%%%%%%%%%%%%%%%%%%%%%%%%%%%%%%%%%%%%%%%%%%%%%%%%%%%%%%%%%%%%%
  \headerbox{Choix du modèle}{name=method,column=0,below=problem,bottomaligned=speed}{
%%%%%%%%%%%%%%%%%%%%%%%%%%%%%%%%%%%%%%%%%%%%%%%%%%%%%%%%%%%%%%%%%%%%%%%%%%%%%%
    \small{Inspirés par les modèles Mandelbrot, Press et Praetz , Madan et Seneta \cite{madan1990variance} présentent un ensemble de caractéristiques essentielles pour un modèle de rendements financiers:}
    \small{\begin{enumerate}
    \item Distribution de $R_t$ ayant une queue longue
    \item Distribution de $R_t$ ayant des moments finis
    \item Processus en temps continu ayant des accroissements stationnaires et indépendants. Distribution des accroissement de même famille peu importe la longueur.
    \item Extension multivariée afin de conserver la validité du modèle CAPM.
    \end{enumerate}}
    \small{Ce tableau décrit le respect des conditions pour les différents modèles étudiés par les auteurs}
\begin{center}
      \begin{tabular}{|l|cccc|}
        \hline
        & \multicolumn{4}{c|}{\scriptsize{Conditions}} \\
        \hline
        \scriptsize{Modèle}     & \scriptsize{1} & \scriptsize{2} & \scriptsize{3}  & \scriptsize{4}  \\
        \hline
        \scriptsize{Mandelbrot} & \scriptsize{X} &   &   & \scriptsize{X} \\
        \scriptsize{Press}      & \scriptsize{X} & \scriptsize{X} & \scriptsize{X} & \scriptsize{X} \\
        \scriptsize{Praetz}     & \scriptsize{X} & \scriptsize{X} &   & \scriptsize{X} \\
        \scriptsize{Madan et Seneta} & \scriptsize{X} & \scriptsize{X} & \scriptsize{X} & \scriptsize{X} \\
        \hline
      \end{tabular}
\end{center}
  }
%%%%%%%%%%%%%%%%%%%%%%%%%%%%%%%%%%%%%%%%%%%%%%%%%%%%%%%%%%%%%%%%%%%%%%%%%%%%%%
  \headerbox{Fonction caractéristique}{name=background model,column=1,below=problem,bottomaligned=speed}{
%%%%%%%%%%%%%%%%%%%%%%%%%%%%%%%%%%%%%%%%%%%%%%%%%%%%%%%%%%%%%%%%%%%%%%%%%%%%%%
    \small{On définit la variable aléatoire $Y$
    \begin{equation*}
      \label{eq:defvarY-GAL}
      Y = \theta + \mu W + \sigma \sqrt{W} Z
    \end{equation*}
    La fonction caractéristique de la distribution de $Y$ peut être
    obtenue en utilisant la formule de l'espérance conditionnelle.
    \begin{align*}
      \phi_Y&(t;\theta,\sigma,\mu,\tau) = E\left[e^{ity}\right] \\
      &= E\left[E\left[e^{ity} | W \right] \right]  \\
      &= \int_0^{\infty} E \left[ e^{it(\theta + \mu w+\sigma\sqrt{w}Z)} \right] g(w) dw \\
      &= e^{i\theta t}\int_0^{\infty} e^{ i\mu w t-\frac{\sigma^2t^2w}{2}} \times \frac{1}{\Gamma (\tau)} w^{\tau-1}e^{-w} \\
      &= \frac{e^{i\theta t}}{\left(1+\frac{1}{2} \sigma^2 t^2 - i\mu t \right)^{\tau}}
    \end{align*}
    On retrouve la fonction de répartition par le théorème de Gil-Pelaez
    \begin{equation*}
      \label{eq:inversionfncaract}
      F_Y(y) = \frac{1}{2} - \frac{1}{\pi}\int_{0}^{\infty} \frac{Im\left[e^{-ity}\phi_Y(t)\right]}{t} dt
    \end{equation*}}
  }
\end{poster}

\end{document}
